\section{Conclusion}
\label{conclusion}

In this paper, we introduce the \emph{partial enclosure range searching 
problem}.
Two variants of the problem are studied. In the first variant, 
a set of line segments $S$ is preprocessed so that the partial enclosure range
query for a query range $Q$ can be performed efficiently. In the second 
variant, 
$S$ is a polygon and $Q$ is an axis-parallel rectangle, and the objective of 
the 
\emph{partial enclosure area problem} is to compute the area of $S \cap Q$. 

When $S$ is a monotone polygon, our presented algorithm requires $O(n\log n)$ 
preprocessing 
time and space and the query time is $O(\log n)$. In \cite{BINT}, it is shown 
that the space can be 
improved to $O(n)$ by increasing the query time to $O(\sqrt{n})$. It is also 
shown in \cite{BINT} that 
if $S$ is a convex polygon then, in $\BigOh{n}$ time and space, we can create a 
data structure which 
can compute the area of $S \cap Q$ in $\BigOh{\log{n}}$ time, for any arbitrary 
oriented rectangular 
query range $Q$. For the case where $S$ is a simple polygon, we can handle 
queries where $Q$ is an 
axis-parallel slab. Unfortunately, we cannot extend our method for rectangular 
queries to work with simple polygons so easily.
While the multi-region formulas themselves do not use the monotone property, our 
tree of multi-region formulas does. 
The tree functions by partitioning the trapezoidal regions with respect to 
vertical lines, however, in a simple polygon, a vertical line passing through 
the boundary of one region may pass through the interior of another. 
This lack of clean partitioning prevents the multi-region formula from working 
correctly for all possible horizontal query lines which may be given as input to 
the formula. Thus, the partial enclosure problem for simple polygons is worth studying. 
