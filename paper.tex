\documentclass[a4paper,11pt]{article}
\usepackage[lined,commentsnumbered,ruled,noend,
boxed]{algorithm2e}
\usepackage{graphicx,amssymb,amsmath,amsthm,textcomp}
%\input{psfig.sty}
\setlength{\parskip}{1.2mm}
\setlength{\parindent}{0pt}
\setlength{\textheight}{8.95in}
\setlength{\textwidth}{6in}
\newtheorem{theorem}{Theorem}
\newtheorem{lemma}{Lemma}
\newtheorem{prop}{Proposition}
\newtheorem{obsv}{Observation}
\newtheorem{result}{Result}
\newtheorem{defn}{Definition}
\usepackage{color}
\newcommand{\red}[1]{{\textcolor{red}{#1}}}
\newcommand{\blue}[1]{{\textcolor{blue}{#1}}}
\newcommand{\green}[1]{{\textcolor{green}{#1}}}
\newtheorem{problem}{Problem}

\newcommand{\complain}[1]{\textcolor{red}{#1}}
\newtheorem{observation}{Observation}
\newtheorem{definition}{Definition}
\newtheorem{remark}{Remark}
\newtheorem{fact}{Fact}
\newtheorem{property}{Property}
\newtheorem{corollary}{Corollary}[theorem]
\renewcommand{\baselinestretch}{1.17}
\setlength{\oddsidemargin}{-0.1in}
\setlength{\topmargin}{-0.1in}
\newcommand{\remove}[1]{}
\newcommand{\PERS}[0]{PERS}
\newcommand{\PEAC}[0]{PEAC}
\newcommand{\st}{\;|\;}
\newcommand{\imply}{\;\rightarrow\;}
\newcommand{\floor}[1]{\left\lfloor#1\right\rfloor}
\newcommand{\ceil}[1]{\left\lceil#1\right\rceil}
\newcommand{\apriori}{\textit{a priori }}
\newcommand{\eg}{\textit{e.g.}}
\newcommand{\ie}{\textit{i.e.}}

%% Asymptotic notation
\newcommand{\BigOh}[1]{O\!\left(#1\right)}
\newcommand{\LittleOh}[1]{o\!\left(#1\right)}
\newcommand{\BigOmega}[1]{\Omega\!\left(#1\right)}
\newcommand{\LittleOmega}[1]{\omega\!\left(#1\right)}
\newcommand{\BigTheta}[1]{\Theta\!\left(#1\right)}

%% lines and segments
\newcommand{\iline}[1]{\overline{#1}}

%% polygons and area
\newcommand{\area}[1]{\operatorname{area}\!\left(#1\right)}

\title{Partial Enclosure Range Searching\thanks{Research supported by NSERC.}} 

\author{Gregory Bint$^1$ \and Anil Maheswari$^1$ \and
        Subhas C. Nandy$^2$ 
        \and Michiel Smid$^1$ }
\date{$^1$ School of Computer Science, Carleton University, Canada, \\{\tt \{gbint,
anil, michiel}@scs.carleton.ca\}\\
$^2$ Indian Statistical Institute, Kolkata, India,
{\tt  nandysc@isical.ac.in}}
% Add the appropriate index information
%------------------------------ Text -------------------------------------

\begin{document}
\maketitle

{\bf Abstract:}
A new type of range searching problem, called the \emph{partial 
enclosure range searching problem}, is introduced in this paper. 
Given a set of geometric objects $S$ and a query region $Q$, our 
goal is to identify those objects in $S$ which intersect the 
query region $Q$ by at least a fixed proportion of their original 
size. Two variations of this problem are studied. In the first 
variation the objects in $S$ are line segments and the objective 
is to count the total number of members of $S$ so that their 
intersection with $Q$ is at least a given proportion of their size. 
Here $Q$ can be an axis-parallel rectangle or a slab of arbitrary 
orientation. In the second variation, $S$ is a polygon and $Q$ is 
an axis parallel rectangle. 
The problem is to report the area of the intersection between the 
polygon $S$ and a query rectangle $Q$. 
%Here we have studied two subcases depending on whether $S$ is a 
%convex or a monotone polygon.


\section{Introduction}
In a geometric range searching problem, a set of geometric objects 
$S$ are given, such as points, lines, circles, 
or boxes, and the query is with respect to another well-defined 
geometric object $Q$. The objective is to identify all 
elements in $S$ contained within the query region $Q$. 
Traditionally, preprocessing schemes are developed to build a data 
structure so that queries can be answered efficiently. Over past four decades 
several variants of range searching problems have been studied depending 
on the complexity of the objects in $S$, the 
query region $Q$, and the query requirements. 

In this paper, we address a different variation of this problem, 
called \emph{partial enclosure range searching (\PERS{})}. To the 
best of our knowledge, this problem is not studied previously. In 
this setting, the goal is to identify, for a given query region 
$Q$, all objects in $S$ that satisfy the \emph{partial enclosure 
property}. An object in $S$ is said to satisfy the \emph{partial 
enclosure property} if at least some fixed proportion of the object
(with respect to its length, area, volume) must be inside the query 
region $Q$. 

This problem was inspired by the use of Microsoft 
OneNote. Using a digital pen, OneNote can be used much like a paper 
notebook, allowing the user to add handwriting, diagrams, equations, 
etc. to a page. 
Figure~\ref{fig:intro:onenote} shows some handwritten notes, and a 
diagram which has been partially selected. Here the horizontal 
line segments of the diagram are not entirely enclosed by 
the selection tool, but they appear as part of the set of selected 
items. This behaviour of selecting partially enclosed objects is 
described in a patent \cite{lassoselect}. 
With the rising popularity of touch and pen-enabled devices, this 
need is likely to increase. Although the problems that we will 
examine take place in simpler settings, we will nevertheless develop 
an understanding of the major challenges of this problem domain, as 
well as some techniques for addressing them.


\begin{figure}[t]
\begin{center}
  \includegraphics[width=0.50\textwidth]{figures/fig_onenote}
  \caption{An example of the partial enclosure range searching in 
  Microsoft OneNote. The selected line segments are not entirely 
  enclosed in the query region.}
  \label{fig:intro:onenote}
\end{center}
\end{figure}

The paper proposes algorithms for the following variations of 
the partial enclosure range searching problem.  
\begin{description}
\item[\PERS{} - Partial enclosure range searching:] Here $S$ 
is given as a set of line segments, and the query object $Q$ 
is an axis-parallel rectangle or a slab bounded by two parallel 
lines of arbitrary orientation. The objective is to count 
the number of objects in $S$ that partially/fully lie in $Q$. We have 
considered different combinations depending on the orientation 
of the line segments in $S$ and the orientation of the 
rectangle/slab $Q$.
\item[\PEAC{} - Partial enclosure area computation:] Here $S$ 
is a monotone or arbitrary polygon, and $Q$ is an axis-parallel 
rectangle or slab respectively. The objective is to compute the 
area of the region $S\cap Q$. 
\end{description}

Table~\ref{tab:contributions} gives a broad overview of the 
proposed results in this paper. Here, `AP' is used for {\it 
Axis-Parallel}, `AO' for {\it Arbitrary Orientation}, and 
`P' for {\it Polygon}.

\begin{table}[t]
\caption{Summary of Contributions}
\label{tab:contributions}
\centering
\begin{tabular}{l l l l l l}
\hline \hline
Problem & Object & Query  & Space & Time & Query time\\
\hline
\PERS{}& AP Segment & AP Rectangle  & $O(n\log^3n)$ & $O(n\log^3n)$ & $O(\log^3n$ \\
\PERS{} & AO Segment & AP Rectangle  & $O(n\log^7n)$ & $O(n\log^7n)$ & $O(\sqrt{n}\log^7n)$ \\
\PERS{} & AP Segment & AO Slab  & $O(n\log^2n)$ & $O(n\log^3n)$ & $O(\sqrt{n}\log^3n)$ \\
\PERS{} & AP Segment & 2 AO Slabs  & $O(n\log^3n)$ & $O(n\log^3n)$ & $O(\sqrt{n}\log^3n)$ \\
%\PEAC{} & Convex P & Rectangle  & $O(n)$ & $O(n)$ & $O(\log n)$ \\
%\PEAC{} & Convex P & Convex $k$-gon  & $O(n)$ & $O(n)$ & $O(k \log n)$ \\
\PEAC{} & Monotone P & AP Rectangle  & $O(n\log n)$ & $O(n\log n)$ & $O(\log n)$ \\
%\PEAC{} & Monotone P & AP Rectangle  & $O(n)$ & $O(n\log n)$ & $O(\sqrt{n})$ \\
\PEAC{} & Simple P & Horiz Slab  & $O(n)$ & $O(n)$ & $O(\log n)$ \\
\hline
\end{tabular}
\end{table}


The next four sections of the paper cover partial enclosure 
range searching queries on successively more sophisticated 
geometric objects. In Section \ref{prelim}, we state some 
preliminaries. Section~\ref{:rectangles} focuses on the 
partial enclosure range counting problem when the objects 
are line segments and the query region is an axis-parallel 
rectangle, Section~\ref{:slabs} considers the counting problem 
in the same environment where the query region is an 
arbitrary-oriented slab. Section 
\ref{:monotonep} considers the partial enclosure area 
computation problem in polygons. Finally, we 
conclude in Section~\ref{conclusion}, where we summarize our 
contributions and future work. 

\section{Preliminaries}
\label{prelim}

Let $P$ be a set of $n$ points in $I\!\!R^d$ ($d \geq 2$). A range tree for $P$ 
is a data structure that supports counting and reporting the members of $P$ in 
an axis-parallel rectangular query range
\cite[Chapter~5]{Deberg}.
It can be constructed in  $\BigOh{n \log^{d-1}{n}}$ time using 
$\BigOh{n \log^{d-1}{n}}$ space. The counting and reporting query time 
complexities are 
$\BigOh{\log^{d-1}{n}}$ and $\BigOh{\log^{d-1}{n} + k}$ respectively, where  $k$ 
is the number of reported points.

The range trees are used to query ranges which can be expressed by a single 
query variable expression.
In this paper, in some cases our ranges will take the form of half-planes 
comprised of two query variables, 
and for those, we use the following structure described in Chan\cite{chan2012}, 
restated here with $d=2$.

\begin{theorem}[Corollary 7.3, part $i$, in Chan\cite{chan2012}]
\label{th:chan}
Consider a set $P$ of $n$ points in the plane. We can form $\BigOh{n}$ canonical subsets of total size $\BigOh{n \log{n}}$ in 
$\BigOh{n \log{n}}$ time, such that the subset of all points inside any query 
simplex can be reported as a union of disjoint canonical subsets $C_i$ with 
$\sum_i{\sqrt{|C_i|}} \leq \BigOh{\sqrt{n}\log{n}}$ in time 
$\BigOh{\sqrt{n}\log{n}}$ with high probability with respect to $n$.
\end{theorem}

This structure is particularly well-suited to multi-part queries where we need 
to identify objects which satisfy a half-plane query, 
and which also satisfy some other arbitrary query condition for which we have an 
existing query method.
To support such a query, we construct a canonical subsets structure to perform 
the half-plane component of the query. 
Then, with the objects of each canonical subset, we construct secondary query 
structures.
A similar structure with slightly worse query time bounds by 
Matousek\cite{Matousek92} is presented in \cite[Chapter~16]{Deberg}, 
along with a method for using it to construct multi-part queries.

A multi-level query is executed by first querying the canonical subsets 
structure for all points satisfying the half-plane.
With each subset identified by the first step, we then evaluate the associated 
secondary structure.
The result is the set of all objects which satisfy both the half-plane query 
\emph{and} whatever conditions the second-level query may test.
If the secondary structure is itself a multi-level structure, then this 
procedure effectively adds one more layer, and we can repeat this process to any 
arbitrary number of levels.
The following corollary formalizes the use of canonical subset structures to 
build multi-level structures and details the preprocessing and query 
requirements.


\begin{corollary}
\label{cor:multichan}

Given a set of $n$ objects $a_1, a_2, \ldots, a_n$ which can be queried with a 
data structure $A$ requiring preprocessing space $S(n)$, 
preprocessing time $T(n)$, and query time $Q(n)$, and where each $a_i$ also has 
an associated point $p_i$ in the plane, we can construct a multi-level 
data structure which can identify all objects whose associated point $p_j$ 
satisfies a half-plane  \emph{and} where $a_j$ satisfies a query on $A$.

\begin{enumerate}
\item If $S(n) \in \BigOh{n \log^f{n}}$, $T(n) \in \BigOh{n \log^g{n}}$, and 
$Q(n) \in \BigOh{\sqrt{n} \log^h{n}}$, where $f, g, h \in \BigOh{1}, 0 \leq f 
\leq g$, then the resulting multi-level data structure requires 
$\BigOh{n\log^{f+1}{n}}$ preprocessing space, $\BigOh{n\log^{g+1}{n}}$ 
preprocessing time, and $\BigOh{\sqrt{n}\log^{h+1}{n}}$ query time with high 
probability.

\item If $S(n) \in \BigOh{n \log^f{n}}$, $T(n) \in \BigOh{n \log^g{n}}$, and 
$Q(n) \in \BigOh{\log^h{n}}$, where $f, g, h \in \BigOh{1}, 0 \leq f \leq g$, 
then the resulting multi-level data structure requires $\BigOh{n\log^{f+1}{n}}$ 
preprocessing space, $\BigOh{n\log^{g+1}{n}}$ preprocessing time, and 
$\BigOh{\sqrt{n}\log^{h+1}{n}}$ query time with high probabily.

\end{enumerate}
\end{corollary}

\begin{proof}
The preprocessing space of the multi-level structure requires $\BigOh{n 
\log{n}}$ space for the canonical subsets structure itself. In addition, the 
associate structures need 
$\sum_{i=1}^k{S(|C_i|)}$ space, which is less than or equal to 
$\sum_{i=1}^k{\BigOh{|C_i| \log^f{|C_i|}}}$ 
$\leq \BigOh{\log^f{n} \cdot \sum_{i=1}^k{|C_i|}}$ (since $|C_i| \leq n$ for all 
$i$)  
$\leq \BigOh{\log^f{n} \cdot n \log{n}}$  (by Theorem \ref{th:chan}). 
Thus, the total space complexity is $\BigOh{n\log^{f+1}{n}}$.
Preprocessing time is calculated in the same manner, resulting in the time 
complexity  of $\BigOh{n\log^{g+1}{n}}$.

Querying requires $\BigOh{\sqrt{n}\log{n}}$ time with high probability  to find 
the $k'$ disjoint canonical 
subsets representing the elements found by the top-level canonical subsets 
query, plus the time required to 
query the associated data structures.  

If $Q(n) \in \BigOh{\sqrt{n}\log^h{n}}$ then the total time to query the 
appropriate associated structures is
$\sum_{i=1}^{k'}{Q(|C_i|)}$
$\leq \sum_{i=1}^{k'}{\BigOh{\sqrt{|C_i|}\log^h{|C_i|}}}$
$\leq \BigOh{\log^h{n} \cdot \sum_{i=1}^{k'}{\sqrt{|C_i|}}}$
$\leq \BigOh{\log^h{n} \cdot \sqrt{n}\log{n}}$ (by Theorem \ref{th:chan}).

Otherwise, if $Q(n) \in \BigOh{\log^h{n}}$ then the total time to query the 
appropriate associated structures is
$\sum_{i=1}^{k'}{Q(|C_i|)}$
$\leq \sum_{i=1}^{k'}{\BigOh{\log^h{|C_i|}}}$
$\leq \BigOh{\sum_{i=1}^{k'}{\log^h{n}}}$
$\leq \BigOh{\log^h{n} \cdot \sum_{i=1}^{k'}{1}}$
$\leq \BigOh{\log^h{n} \cdot \sqrt{n}\log{n}}$ (by Theorem \ref{th:chan}).

Thus, the total query time  is $\BigOh{\sqrt{n}\log^{h+1}{n}}$ with high 
probability.
\end{proof}

We use this theorem and corollary as ``black boxes'' in Sections 
\ref{:rectangles} and \ref{:slabs}.
\section{PERS problem for Axis-Parallel Rectangles}
\label{:rectangles}


We will present two variations of the \PERS{} problem in this section. 
In the first variation (Subsection~\ref{:rectangles:ap}), the line 
segments in $S$ will  be axis-parallel, whereas in the next variation 
(Subsection~\ref{:rectangles:ao}), we allow the segments in $S$ of 
arbitrary orientation. In both the variations, the query rectangle $Q$ 
is axis-parallel. 

%------------------------------------------------------------------------------
\subsection{Axis-Parallel Segments}
\label{:rectangles:ap}

We start our discussion with only horizontal segments. The solution 
for vertical segments is identical. We also discuss how to combine 
the two approaches at the end of the section. We define the problem 
as follows.

\begin{definition}
A segment $s \in S$ is said to satisfy the {\em partial enclosure 
property} with respect to a query object $Q$ if and only if 
$|s \cap Q| \geq \rho \cdot |s|$, where $|x|$ denotes the length 
of the segment $x$.
\end{definition}

\begin{problem}
Given a set $S$ of $n$ axis-parallel line segments in the plane, 
and a fixed parameter $\rho$ ($0 < \rho \leq 1$), we want to 
identify those segments which are sufficiently enclosed by 
an axis-parallel query rectangle $Q$ so as to satisfy the 
partial enclosure property. 
\end{problem}

We use the following notations during our discussions. Each segment 
$s_i =[(a_i,b_i), \ell_i] \in S$, $1 \leq i \leq n$ is defined 
by its left or bottom endpoints $(a_i, b_i)$ depending on whether 
$s_i$ is horizontal or vertical, and its length $\ell_i$. 
\complain{The tuple format $s_i =[(a_i,b_i), \ell_i]$ is inconsistent with the Slabs section}
Our query rectangle $Q$ is given by its bottom-left corner $(\alpha, 
\beta)$ and its top-right corner $(\gamma, \delta)$. \remove{Thus, its 
width is $w=\gamma-\alpha$.} We say that $s_i \in_\rho Q$ if and only 
if $s_i$ satisfies the partial enclosure property w.r.t. $Q$, 
otherwise $s \not \in_\rho Q$. 



\begin{figure}[t]
\centering
\includegraphics[width=1.00\columnwidth]{figures/fig_oo_cases}
\caption{An axis-parallel query on axis-parallel segments. Different cases 
of horizontal segments interacting with the query region are shown.}
\label{:fig:rectangles:ap:cases}
\end{figure}

Needless to mention, here we need to consider only the segments 
satisfying $\beta \leq b \leq \delta$. 
Figure~\ref{:fig:rectangles:ap:cases} illustrates several cases 
regarding how a horizontal segment $s \in S$ (satisfying $\beta 
\leq b \leq \delta$) may interact with $Q$. 
Cases $(1)$, $(2)$, and $(3)$ demonstrate the cases where $s$ is 
entirely left, entirely within, or entirely right of $Q$, 
respectively. Case $(4)$ considers the situation where $s$ crosses 
only the left boundary of $Q$ (i.e., $\alpha \leq a+\ell \leq \gamma$). 
Depending on the partial enclosure parameter $\rho$, we further 
subdivide case $(4)$ into subcases $(4a)$ if $s \in_\rho Q$, and 
$(4b)$ if $s \not \in_\rho Q$. Cases $(5a)$ and $(5b)$ are similar 
to cases $(4a)$ and $(4b)$, but with respect to $\gamma$. Specifically, 
$s$ falls into case $(5a)$ or $(5b)$ when $\alpha \leq a \leq \gamma$ 
and $a+\ell > \gamma$. In case $(6)$, $s$ crosses both the left and right 
boundaries of $Q$, with neither of its endpoints inside $Q$;
the subcases are $(6a)$ if $s \in_\rho Q$ and $(6b)$ if $s \not 
\in_\rho Q$.
Our goal is to identify all segments belonging to cases $(2)$, $(4a)$, $(5a)$, and $(6a)$, and none of the segments belonging to any other case.

Let $w = \gamma - \alpha$ be the width of $Q$. Thus, we need to consider 
only the segments in $S_1 = \{s_i \in S \st \beta \leq b_i \leq \delta 
~\&~ \ell_i \leq \frac{w}{\rho}\}$, discarding all segments in case $(6b)$, among others.
We partition the members in $S_1$ 
according the location of their left endpoint with respect to $\alpha$. 
Specifically, let $S_L = \{s_i \in S_1 \st a_i < \alpha\}$ and let 
$S_R = \{s_i \in S_1 \st a_i \geq \alpha\}$.
Now, we test an appropriate partial enclosure expression to determine 
whether $s_i$ should be counted. For segments in $S_L$, we want to 
ensure that ``not too much of $s_i$ is outside of $Q$'', i.e., $S_L' 
= \{ s_i \in S_L \st \alpha - a_i < (1 - \rho) \cdot l_i\}$.
For segments in $S_R$, we want to ensure that ``enough of $s_i$ is inside 
$Q$'', i.e., $S_R' = \{s_i \in S_R \st \gamma - a_i \geq  \rho l_i\}$.

\begin{observation}
The subset of segments satisfying the partial enclosure property is 
$S_\rho = S_L' \cup S_R'$. 
\end{observation}

Thus, the points in $S_L'$ can be identified by mapping each 
segment $s_i\in S$ to a point $\hat{s_i}=(b_i,\rho\cdot\ell_i,
a_i,a_i + (1-\rho) \cdot\ell_i)$ in $I\!\!R^4$, and then observing  
those points lying inside the  four dimensional query box 
$\hat{Q}=[\beta, \delta] \times (0,w] \times (-\infty,\alpha] \times 
(\alpha,\infty)$. 
Similarly, the points in $S_R'$ can be identified 
by mapping each segment $s_i\in S$ to a point 
$\hat{\hat{s_i}}=(b_i,\rho\cdot\ell_i,a_i,a_i + \rho 
\cdot\ell_i)$ in $I\!\!R^4$, and then observing  
those points lying inside the  four dimensional query box 
$[\beta, \delta] \times (0, w] \times [\alpha, \infty) 
\times (-\infty, \gamma]$.

We can answer these queries by constructing two 4D range 
trees~\cite{Deberg} with two sets of points $\{\hat{s_i}|s_i 
\in S\}$ and $\{\hat{s_i}|s_i \in S\}$ respectively, and 
executing the counting query with the corresponding 4D query 
rectangle. The preprocessing time and space required for 
constructing these two range trees are both $\BigOh{n \log^3{n}}$, 
and the query can be answered in $\BigOh{\log^3{n}}$ time. Finally, 
we report the sum of two results as the answer of the query. 
Note that, we can use the same first three levels for both the 
range trees since the first three components of both the types 
of query points (for $a<\alpha$ and $a > \alpha$) are same. In the  
third level, we create \emph{two} associated structures, one for 
each of the partial enclosure expressions, and query each one as 
needed. 

For vertical segments, the method of querying is exactly similar 
to that for horizontal segments, only we need to consider the 
height of $Q$ instead of its width, and need to consider symmetric 
coordinates of each segment while mapping them to points in 4D. 
The following theorem summarizes the solution.

\begin{theorem}
\label{th:ap}
Given a set of $n$ axis-parallel line segments, we can identify a set of disjoint subsets containing all segments which satisfy the partial enclosure property for an axis-parallel query rectangle in $\BigOh{\log^3{n}}$ time, with a data structure requiring $\BigOh{n\log^3{n}}$ preprocessing time and space.
\end{theorem}

%------------------------------------------------------------------------------
\subsection{Arbitrarily-Oriented Segments}
\label{:rectangles:ao}

In this subsection, we allow each segment $s_i \in S$ to have any 
arbitrary orientation. They may intersect among themselves. We use 
$s_i= [(a_i,b_i), (c_i,d_i)]$ to denote a segment, where $(a_i,b_i)$ 
and $(c_i,d_i)$ are two end-points of $s_i$ satisfying $a_i\leq c_i$, 
and if $a_i = c_i$ then $b_i < d_i$. The problem statement is as 
follows.

\begin{problem}
Given a set of $n$ arbitrarily-oriented line segments in the plane, and a 
fixed parameter $\rho$ such that $1/2 < \rho \leq 1$, we want to identify 
those segments which are sufficiently enclosed by an axis-parallel query 
rectangle $Q$ so as to satisfy the partial enclosure property. A segment 
$s$ satisfies this property if and only if $|s \cap Q| \geq \rho \cdot |s|$.
\end{problem}



\subsubsection{Decomposing the Problem}
\label{:rectangles:ao:approach}

We have three principal cases to consider: (i) those segments which 
have both endpoints inside $Q$, (ii) those with only one endpoint inside 
$Q$, and (iii) those with both endpoints outside $Q$ but they intersect $Q$.  

We first construct a 4D range tree $\cal T$ with the points $(a_i, b_i, c_i, d_i)$ 
for each $s_i \in S$, and execute a 4D range query with query box 
$[\alpha, \gamma] \times [\beta, \delta] \times [\alpha, \gamma] \times 
[\beta, \delta]$ corresponding to $Q=[\alpha,\beta] \times [\gamma,\delta]$ 
to identify all the segments satisfying the different cases. 
The points in case (i) will all satisfy the partial enclosure property.

For the segments satisfying case (ii), we apply the same method as in 
case (i), but on a {\it virtual} segment as follows. For each $s_i=
[(a_i,b_i),(c_i,d_i)] \in S$, create two virtual segments $s_i'$ and 
$s_i''$, where $s_i' \subseteq s_i$, $s_i'$ has an endpoint $(a_i,b_i)$, 
and $|s_i'| = \rho \cdot |s_i|$, and $s_i'' \subseteq s_i$, $s_i''$ 
has an endpoint $(c_i,d_i)$, and $|s_i''| = \rho \cdot |s_i|$. Thus, if 
$s_i' \in Q$ and/or $s_i'' \in Q$, then $s_i \in_\rho Q$. Counting both 
cases may double-count some 
segment(s) which has both endpoints in $Q$, so we subtract the count 
obtained in Case (i). 
As like the previous case, we can solve this case using a 4D range query.

Case (iii), where both the endpoints of a segment are outside $Q$, 
is the most challenging case to handle. We begin by partitioning the 
space outside $Q$ into 8 regions by extending lines through its 
horizontal and vertical boundaries. These regions are labelled as 
$I, II, \ldots, VIII$ in anticlockwise order starting from the 
left-middle region (see Figure~\ref{fig:rectangles:ao:regions}). Any 
segment which passes through $Q$ but has its neither endpoint in $Q$ 
will have its endpoints in two distinct regions. Not every pair of 
regions is legal\footnote{A segment with its two endpoints 
in regions $I$ and $II$ respectively, cannot intersect $Q$.}.

\begin{figure}[t]
\begin{center}
  \includegraphics[width=0.5\textwidth]{figures/fig_ao_regions}
  \caption{The 8 regions surrounding $Q$.}
  \label{fig:rectangles:ao:regions}
\end{center}
\end{figure}

A segment which passes through $Q$ may involve in any one of the following 
four cases depending on the pair of regions containing its two endpoints:

\begin{description}
\item[(A)] In two parallely opposite cells: $(I, V), (III, VII)$.
\item[(B)] In two non-corner cells adjacent to a corner of $Q$: $(I, III), (III, V), (V,VII), (VII, I)$.
\item[(C)] In a corner cell and a non-corner cell: $(II, V)$, $(II, VII)$, $(IV, I)$, 
$(IV, VII)$, $(VI, I)$, \newline $(VI, III)$, $(VIII, III)$, $(VIII, V)$.
\item[(D)] In two corner cells: $(II, VI), (IV, VIII)$.
\end{description}

We will query for each class separately and combine our results at the end. 
In the first phase, we identify each class of problem 
by performing an appropriate rectangular range searching in the 
data structure $\cal T$. 
This also indicates which partial enclosure expressions we need to test in 
the second phase of the query. We develop expressions for each case below. 
In all the cases, we require the following additional definitions. Let 
$o_i = (e_i, f_i)$ be the mid-point of $s_i$. We use  
$d(u,v)$ to denote the Euclidean distance between two points $u,v \in 
\mathbb{R}^2$. 

\begin{figure}[h]
\begin{minipage}[b]{0.5\linewidth}
\centering
\includegraphics[width=0.80\textwidth]{figures/fig_ao_case1}\\
(a)
\end{minipage}
\begin{minipage}[b]{0.5\linewidth}
\centering
\includegraphics[width=0.80\textwidth]{figures/fig_ao_case2}\\
(b)
\end{minipage}
\caption{Demonstration of (a) Case A and (b) Case B}
\label{fig:rectangles:ao:case12}
\end{figure}



%------------------------------------------------------------------------------
{\bf Case (A):}
This case deals with the segments that cross the parallel sides of $Q$. 
We present the details of the partial enclosure property for the subcase 
where $p=(a,b) \in I$ and $q=(c,d) \in V$; the solution for the subcase 
$(III, VII)$ is similar.

Assume for now that $o \in Q$. Let 
$p'$ and $q'$ be the points of intersection of the segment $s$ with 
$Q$ closest to $p$  and $q$ respectively. Let us draw a right angle 
triangle $\Delta p u o$ whose edge $[u,o]$ intersects the boundary 
of $Q$ at $u'=(\alpha, f)$. Similarly, the edge $[v,o]$ of  
$\Delta q v o$  intersects the boundary of $Q$ at 
$v' = (\gamma, f)$ (see Figure~\ref{fig:rectangles:ao:case12}(a)). 
The segment $s$ satisfies partial enclosure property if 
$\frac{d(p, p') + d(q, q')}{d(p, q)} \leq 1 - \rho$. By similarity 
of triangles $\Delta puo$ and $\Delta p'u'o$, we have
$\frac{d(p, p')}{d(p, o)} = \frac{d(p, p')}{L} = 
\frac{d(u, u')}{d(u, o)} = \frac{2 d(u, u')}{2 d(u, o)} = 
\frac{2(\alpha - a)}{c - a}$. Similarly, from the similarity of 
triangles $\Delta qvo$ and $\Delta q'v'o$, we have
$\frac{d(q, q')}{d(q, o)} = \frac{2(c - \gamma)}{c - a}$.

Thus, $\frac{d(p,p')+d(q,q')}{d(p,q)}=\frac{d(p,p')+d(q,q')}{2L} = 
\frac{1}{2} \left ( \frac{d(p,p')}{L} + \frac{d(q,q')}{L} \right ) 
 = \frac{\alpha - a}{c - a} + \frac{c - \gamma}{c - a} 
= 1 + \frac{\alpha - \gamma}{c - a}$. 

Therefore, the inequality $\frac{d(p, p') + d(q, q')}{d(p, q)} 
\leq 1 - \rho$ can instead be evaluated as $1 + \frac{\alpha - 
\gamma}{c - a} \leq 1 - \rho$, which is based on two query 
variables $\alpha$ and $\gamma$. We can further simplify the 
expression to $\rho(c - a) \leq \gamma - \alpha$, where the value 
of $\gamma - \alpha$ is a single query variable expression 
calculated at query time. This inequality can therefore be checked 
by augmenting the search tree $\cal T$ with another level with 
respect to the variable $h=\rho(c-a)$ corresponding to the segments 
in $S$, and querying with $h \leq \gamma-\alpha$ at the relevant 
nodes in that level of $\cal T$. Note that, of $o \not\in Q$, then 
$s$ does not satisfy the partial enclosure property, and this test 
will also fail. So, no test is required to check whether $q \in Q$ 
or not.  



%------------------------------------------------------------------------------
{\bf Case (B)} This case is concerned with the segments which cross 
the mutually perpendicular sides of $Q$. We present the details 
of the partial enclosure property for the case where $p \in I$ 
and $q \in III$; the solutions for the other subcases  
are similar. 

Again let us assume that $o \in Q$\footnote{If $o \not\in Q$, $s$ 
will not satisfy the partial enclosure property, and it can be shown 
that the test derived for this case will fail.}. 
As earlier, let $p'$ 
and $q'$ be the points of intersection of $s$ with the boundary of 
$Q$ closest to $p$ and $q$ respectively. As in Case (A), here also 
we draw the 
right-angle triangles $\Delta p u o$ and  $\Delta p'u'o$. In 
$\Delta p u o$, let $u' = (\alpha, f)$ be the point of intersection 
of the line segment $[u, o]$ with the boundary of $Q$, and 
$\Delta p'u'o$ is similar to $\Delta p u o$. Similarly, in 
$\Delta q v o$, the point $v' = (e, \beta)$ 
is the intersection point of $[v, o]$ with the boundary of $Q$, 
where $\Delta q' v' o$ is similar to $\Delta q v o$. See 
Figure~\ref{fig:rectangles:ao:case12}(b). We need to identify those 
segments satisfying $\frac{d(p, p')+d(q, q')}{d(p, q)} \leq 1-\rho$. 

By similarity of triangles, we have  $\frac{d(p, p')}{d(p, o)} = 
\frac{d(p, p')}{L} = \frac{d(u, u')}{d(u, o)} = \frac{2(\alpha - a)}
{c - a}$, and similarly, $\frac{d(q, q')}{d(q, o)} = \frac{2(\beta - d)}{b - d}$. 
Thus,
$\frac{d(p, p') + d(q, q')}{d(p, q)}=\frac{d(p, p') + d(q, q')}{2L} 
=\frac{1}{2} \left (\frac{d(p, p')}{L} + \frac{d(q, q')}{L} \right) 
= \frac{\alpha - a}{c - a} + \frac{\beta - d}{b - d}$.

Therefore, the inequality $\frac{d(p, p') + d(q, q')}{d(p, q)} 
\leq 1 - \rho$ can be tested by checking the equivalent inequality 
$\frac{\alpha - a}{c - a} + \frac{\beta - d}{b - d} \leq 1 - \rho$, 
defined on the two query variables $\alpha$ and $\beta$. On 
simplification, it becomes $\alpha + \beta \cdot \frac{c-a}{b-d} \leq 
\left ( (1 - \rho) + \frac{d}{b-d} \right ) \cdot (c-a) + a$.

Thus, we can map each segment $s$ satisfying Case (B) to a point in the dual plane
given by
\[ (z,w)=
\left(\frac{c-a}{b-d}, \left ( (1 - \rho) + \frac{d}{b-d} \right ) \cdot (c-a) + a \right )
\]
The segments matching the 
partial enclosure expression correspond to the points satisfying the 
half-plane $y \geq \beta x + \alpha$ in the dual plane. In order 
to perform this test, we need to add a half-plane query data structure
(ham-sandwich cut tree) \cite{chan2012} with the points $(z,w)$ 
corresponding to the 
points in $S$ at the leaf level of $\cal T$ (independent of the arrangement 
made in Case (A)), and querying with the halfplane obtained from the query 
interval $Q$ as the desired node in the leaf level of $\cal T$. 

%------------------------------------------------------------------------------
{\bf Case (C)}
In this Case, each segment have one endpoint in a corner region of $Q$, 
and the other endpoint is in a non-corner region of $Q$ as shown in Figure~\ref{fig:rectangles:ao:case3}. We present the details of the partial 
enclosure property for the subcase where $p \in I$ and $q \in IV$; the 
solutions for other subcases of this case are similar.

\begin{figure}[t]
\begin{center}
  \includegraphics[width=0.45\textwidth]{figures/fig_ao_case3}
  \hspace{1.0em}
  \includegraphics[width=0.45\textwidth]{figures/fig_ao_case3b}
  \caption{Example of segments in  subcase $(I, IV)$ of case 
  (C): the blue follows the proper handling, while the red shows the 
  incorrect handling.}
  \label{fig:rectangles:ao:case3}
\end{center}
\end{figure}

Consider the situation where $o \in Q$. We need to consider two 
subcases: (i) $s$ crosses the lines $x=\alpha$ and $y=\beta$ and 
(ii) $s$ crosses the lines $x=\alpha$ and $x=\gamma$. See 
Figures~\ref{fig:rectangles:ao:case3}(a) and 
\ref{fig:rectangles:ao:case3}(b) for examples. Subcase (i) is very 
close to the example presented for case (B). In that case, 
the only use of the fact that the endpoints of $s$ were located in 
regions $I$ and $III$ was to imply that $s$ crossed the lines 
$x=\alpha$ and $y=\beta$. As such, this subcase can use the same 
expression for testing the partial enclosing property for the segment 
$s$. Likewise, Subcase (ii) is similar to the 
example presented for case (A) and can use exactly that expression 
for testing $s$.

Our initial query for identifying the regions of $p$ and $q$ does 
not allow us to differentiate between the subcases. Instead, we will 
check both subcases simultaneously. From case (A) and case (B), 
we have the following expressions:
$ \frac{\alpha - a}{c - a} + \frac{c - \gamma}{c - a} \leq 1 - \rho$
 and
$\frac{\alpha - a}{c - a} + \frac{\beta - d}{b - d} \leq 1 - \rho$
respectively. If $s$ belongs to subcase (i), then 
$\frac{c - \gamma}{c - a} \leq \frac{\beta - d}{b - d}$ since 
$\gamma$ is farther right than the point where $s$ exits from $Q$. 
Likewise, in subcase (ii), 
$\frac{\beta - d}{b - d} \leq \frac{c - \gamma}{c - a}$.
Therefore, in either subcase, the result of the correct expression 
is larger than the incorrect one, allowing us to correctly reject 
segments by blindly checking both the conditions. This also holds when 
$o \not \in Q$, since the expression for at least one subcase would 
exclude the segment.


% -----------------------------------------------------------------------------
{\bf Case (D)}
This case is concerned with segments whose endpoints appear in 
diagonally opposite corner regions of $Q$. We present the details 
of the partial enclosure property for the case where $p \in II$ 
and $q \in VI$; the solution for the subcase $(VI, VIII)$ is similar.

\begin{figure}[t]
\begin{center}
  \includegraphics[width=0.45\textwidth]{figures/fig_ao_case4a}
  \hspace{1.0em}
  \includegraphics[width=0.45\textwidth]{figures/fig_ao_case4b}

  \vspace{2.0em}
  
  \includegraphics[width=0.45\textwidth]{figures/fig_ao_case4c}
  \hspace{1.0em}
  \includegraphics[width=0.45\textwidth]{figures/fig_ao_case4d}

  \caption{Example of segments of subcase case $(II, VI)$ of case (D): 
  in each figure, the blue lines show the proper handling.}
  \label{fig:rectangles:ao:case4}
\end{center}
\end{figure}

Assume for now that $o \in Q$. As in case (C), here also we need to 
consider several subcases depending on which sides of $Q$ are 
intersected by $s$, namely (i) $s$ crosses the lines $x=\alpha$ and 
$y=\delta$, (ii) $s$ crosses the lines $x=\alpha$ and $x=\gamma$, 
(iii) $s$ crosses the lines $y=\beta$ and $y=\delta$, and (iv)
$s$ crosses the lines $y=\beta$ and $x=\gamma$ (see Figure~
\ref{fig:rectangles:ao:case4}(a-d) for example of each subcase. 

Our query in the first phase for identifying the regions of $p$ and $q$ 
does not allow us to differentiate between subcases. However, 
it is sufficient to check all subcases simultaneously. The 
expression for subcase (ii) comes directly from case (A).
Using similar techniques as we did for cases (A) and (B), 
we develop the following set of partial enclosure expressions:

\begin{align*}
& (i) \quad \; \; \frac{\alpha - a}{c - a} + \frac{d - \delta}{d - b} \leq 1 - \rho
& (ii) \quad \frac{\alpha - a}{c - a} + \frac{c - \gamma}{c - a} \leq 1 - \rho \\
& (iii) \quad \frac{\beta  - b}{d - b} + \frac{d - \delta}{d - b} \leq 1 - \rho  
& (iv) \quad \frac{\beta  - b}{d - b} + \frac{c - \gamma}{c - a} \leq 1 - \rho \\
\end{align*}

Subcases (ii) and (iii) can be simplified to orthogonal range 
queries, while in subcases (i) and (iv) each segment needs to be mapped to 
a point in an appropriate plane and, as in case 
(B), the halfplane counting query needs to be performed for answering 
the query problem. 

To illustrate that checking all four conditions simultaneously 
yields correct results, consider what happens if $s$ belongs 
to subcase (i), where $s$ crosses $\alpha$ and $\delta$. In 
these circumstances, $u'$ is above $\beta$, and we have that 
$\frac{\alpha - a}{c - a} \geq \frac{\beta - b}{d - b}$.  
Likewise, $v'$ is left of $\gamma$, so $\frac{d - \delta}{d - b} 
\geq \frac{c - \gamma}{c - a}$.  Therefore, the expression for 
(i) dominates the other expressions (i.e., $\text{(a)} \geq 
\text{(b)}$, $\text{(a)} \geq \text{(c)}$, and $\text{(a)} \geq 
\text{(d)}$), allowing us to identify qualifying segments for 
subcase (i) by blindly checking all conditions. This property 
is true for segments belonging to subcases (ii), (iii), and (iv) 
as well. Furthermore, this test also holds when $o \not \in Q$, 
since the expression for at least one subcase would exclude the 
segment.


%------------------------------------------------------------------------------
\subsection{Construction and Analysis}
\label{:rectangles:ao:analysis}

Our method takes a very case-by-case approach to solving the problem, 
and executes in two phases. The first phase must classify segments 
belonging one of the interesting classes among (i) to (iv), and then 
its appropriate subclass by searching the data structure $\cal T$.  
In the second phase, the partial enclosure property is tested by 
querying in the respective augmented data structures of $\cal T$ in 
its last level. 

Broadly, our solution to this problem uses a multi-level range tree 
\cite{Deberg} for the classification phase, and a combination of 
range trees and canonical subsets structures \cite{chan2012} to 
check the partial enclosure expressions. 

Identifying the segments which satisfy each case is a matter of 
testing a set of several conditions, all of which must be true. 
The order that we test the conditions in makes no difference to 
the correctness of the algorithm, but can have an impact on its 
space requirements. We now summarize the total time and space 
complexity for solving this problem. 

As mentioned earlier, one can count the segments satisfying the
partial enclosure property among those belonging to the Cases (i) 
and (ii) by performing a 4D range query in the corresponding 
range tree $\cal T$. 

The sticks belonging Case (iii-A) and satisfying the partial enclosure 
property can be counted using 5D range searching with a rectangle $Q'=
(-\infty, \alpha) \times [\beta, \delta] \times (\gamma, \infty) 
\times [\beta, \delta] \times (-\infty, \gamma - \alpha]$ among the 
points $v_i = (a_i, b_i, c_i, d_i, \rho(c_i - a_i))$ corresponding 
to the segments $s_i \in S$. The following lemma summarizes 
the complexity results. 

\begin{lemma}
\label{lem:ao:class1:v}
Given a set $S$ of $n$ line segments, it can be preprocessed 
in $\BigOh{n\log^4{n}}$ preprocessing time and space, and one
can identify a subset of $S$ belonging to Case (iii-A) \emph{and} 
satisfying the partial enclosure property in $\BigOh{\log^4{n}}$ 
time.  
\end{lemma}

In case (iii-B)
\paragraph{Class (B) - subcase $(p, q) \in (I, III)$:} 
As with all cases, part of the problem involves first identifying those segments with endpoints in the appropriate regions.
In this case, however, the partial enclosure expression is evaluated using a half-plane query.

The classification portion is performed just as we have seen above. We map each segment to the 4D point $v_i = ( a_i, b_i, c_i, d_i )$.
The partial enclosure expression will be queried by the dual-space we developed in Section~\ref{:rectanges:ao:class2}. 
For each $s_i \in S$, we define $h_i = \left ( \frac{c - a}{b - d}, \left ( (1 - \rho) + \frac{d}{b-d} \right ) \cdot (c-a) + a \right )$.
We need to construct a data structure which can answer queries on pairs $(v_i, h_i)$.

By Theorem~\ref{th:rangetree}, we can query the classification component using a range tree according to the following lemma.
\begin{lemma}
\label{lem:ao:class2:v}
Given a set of $n$ line segments, we can identify a set of disjoint subsets containing all segments belonging to case (B) in $\BigOh{\log^3{n}}$ time using a data structure requiring $\BigOh{n\log^3{n}}$ preprocessing time and space.
\end{lemma}

By Theorem~\ref{th:chan}, we can query the half-plane component of our query objects using a canonical subsets data structure, giving the following lemma.

\begin{lemma}
\label{lem:ao:class2:h}
Given a set of $n$ line segments, we can identify a set of disjoint subsets containing all segments satisfying a half-plane representation of a partial enclosure expression in $\BigOh{\sqrt{n}\log{n}}$ time, using a data structure requiring $\BigOh{n\log{n}}$ preprocessing time and space.
\end{lemma}

Since the order that we check our conditions in does not affect correctness, we will check the half-plane condition first, then the endpoint classification.  
By Corollary~\ref{cor:multichan}, we can accomplish this by associating a classification structure with each subset of the half-plane structure. The resulting structure is summarized by the following lemma.

\begin{lemma}
\label{lem:ao:class2:c}
Given a set of $n$ line segments, we can identify a set of disjoint subsets containing all segments belonging to Case (B) \emph{and} which satisfy their partial enclosure property in $\BigOh{\sqrt{n}\log^4{n}}$ time using a data structure requiring $\BigOh{n\log^4{n}}$ preprocessing time and space.
\end{lemma}


\paragraph{Case (C) - subcase $(p, q) \in (I, IV)$:} 
This case can use the same orthogonal structure that we developed in Lemma~\ref{lem:ao:class1:v} and the same half-plane expression as in Lemma~\ref{lem:ao:class2:h}. 
One component of the query box needs to be updated to account for region IV, giving the following.
\[
(-\infty, \alpha) \times [\beta, \delta] \times (\gamma, \infty) \times (-\infty, \beta) \times (-\infty, \gamma - \alpha]
\]

By Corollary~\ref{cor:multichan}, we can combine these two data structures to create a new structure which can answer this case as summarized by the following lemma.

\begin{lemma}
\label{lem:ao:class3:c}
Given a set of $n$ line segments, we can identify a set of disjoint subsets containing all segments belonging to class (iii) \emph{and} which satisfy their partial enclosure property in $\BigOh{\sqrt{n}\log^5{n}}$ time using a data structure requiring $\BigOh{n\log^5{n}}$ preprocessing time and space.
\end{lemma}


\paragraph{Case (D) - subcase $(p, q) \in (II, VI)$:} 
This case has the largest number of partial enclosure expressions which need checking, in addition to the usual endpoint classification step. 
As a result, it will be the largest data structure to build.

The basic steps are just as in the last three cases. 
To cover endpoint classification and the two orthogonal partial enclosure expressions, we will use a data structure and query box similar to Lemma~\ref{lem:ao:class1:v}, but extended to 6D to cover the extra partial enclosure expression.
We then apply Lemma~\ref{lem:ao:class2:h} and Corollary~\ref{cor:multichan} twice to handle the two half-plane partial enclosure expressions and associate the orthogonal range tree. 
The entire structure for this case is summarized by the following lemma.

\begin{lemma}
\label{lem:ao:class4:c}
Given a set of $n$ line segments, we can identify a set of disjoint subsets containing all segments belonging to Case (D) \emph{and} which satisfy their partial enclosure property in $\BigOh{\sqrt{n}\log^7{n}}$ time using a data structure requiring $\BigOh{n\log^7{n}}$ preprocessing time and space.
\end{lemma}


\paragraph{Combining the Steps.} 

Querying the entire environment requires us to create the structures for handling the cases where one or both endpoints of a line segment lie entirely inside a query $Q$, as well as the structures for each of the cases when both endpoints lie outside of $Q$. 
Overall, this process is dominated by the structure required for the case (D) segments. 
The following theorem summarizes the overall solution.

\begin{theorem}
\label{th:ao}
Given a set of $n$ arbitrarily-oriented line segments, we can identify a set of disjoint subsets containing all segments which satisfy the partial enclosure property for an axis-parallel query rectangle in $\BigOh{\sqrt{n}\log^7{n}}$ time, using a data structure requiring $\BigOh{n\log^7{n}}$ preprocessing time and space.
\end{theorem}



\input{slabs}
%\section{PEAC Problem on Convex Polygons}
\label{:convexp}

In this section, we consider the partial enclosure area computation 
problem on convex polygons. 
The main contribution of this section is a method to calculate the area of a convex polygon appearing within a query rectangle. 
With the enclosed area in hand, deciding on the partial enclosure property is straight-forward.


%------------------------------------------------------------------------------
\subsection{Problem Definition}
\label{:convexp:problem-definition}

\begin{problem}
We are given a convex polygon $P$ consisting of $n$ edges and a fixed parameter $\rho$ such that $0 < \rho \leq 1$. We want to determine whether at least $\rho \cdot \area{P}$ is enclosed by a query rectangle $Q$.
\end{problem}

Throughout this chapter we use the following definitions. The polygon $P$ is defined by its edges $E = e_1, e_2, \ldots, e_{n-1}, e_n$, given in clockwise order, where $e_n$ and $e_1$ share a vertex to close the polygon.  A \emph{chord} of $P$ is a straight line segment incident to two edges which partitions $P$ into two regions.

A query $Q$ is given by its lower-left and upper-right corners $(\alpha, \beta)$ and $(\gamma, \delta)$, respectively. See Figure \ref{fig:convexp:example} for an example.

\begin{figure}[t]
\begin{center}
  \includegraphics[width=0.50\textwidth]{figures/fig_cvp_example}
  \caption[A convex polygon $P$ and query box $Q$.]{The query region $Q$ formed by the inputs $\alpha$, $\beta$, $\gamma$,  $\delta$, and a convex polygon $P$.}
  \label{fig:convexp:example}
\end{center}
\end{figure}

We say that $P$ is sufficiently enclosed by $Q$ to satisfy the partial enclosure property if and only if $\area{Q \cap P} \geq \rho \cdot \area{P}$. 
Throughout this chapter we will assume that $Q$ is an axis-parallel rectangle for ease of discussion. 
In subsection~\ref{:convexp:remarks} we will show how to use our method even when relaxing some restrictions on $Q$.

%------------------------------------------------------------------------------
\subsection{Overview of the Algorithm}
\label{:convexp:approach}

To achieve a fast query time, our algorithm requires some preprocessing on $P$ to speed up certain area calculations. Before describing the preprocessing, we outline the query algorithm here.

\begin{enumerate}
 \item Identify which edges of $P$ are intersected by $Q$.
 
 \item Enumerate the intersecting edges in clockwise order to determine which areas of $Q$ are inside of $P$, and which are outside. There are at most 4 pairs of intersecting edges where $P$ crosses into $Q$, and subsequently back out.
 
 \item For each entrance/exit pair:
 \begin{enumerate}
  \item Consider the point $s$ on the entrance edge and the point $t$ on the exit edge where $P$ intersects the boundary of $Q$.
  The segment $st$ through the interior of $Q$ forms a \emph{chord} of $P$.
  
  \item On one side of $st$, we have the subpolygon $P_{st}$ defined by a subchain of $P$ (including some partial component of the entrance and exit edges) and the segment $st$ itself. 
  $P_{st} \subseteq Q$, so we must calculate its area as part of our overall query.
  
 \end{enumerate}

 \item Assuming that $P$ and $Q$ \emph{do} intersect, the assembled collection of all such entrance and exit points from the previous step, as well as any corners of $Q$ which are inside $P$ define $P_I$.
We can also think of this as the space on the ``insides'' of all the chords. 
$P_I$ is a polygon of not more than 8 edges. 
$P_I \subseteq Q$, so we must calculate its area as part of our overall query.
 
\end{enumerate}


% -----------------------------------------------------------------------------
\subsection{Preprocessing}
\label{:convexp:preprocessing}

In order to help with calculating the area of the subpolygons identified by step 3(b), above, we will preprocess the area of several chords of $P$, storing the results in a binary search tree $T_P$. 

A chord from edge $e_i$ to $e_j$, $i < j$, is defined as the segment through the interior of $P$ from the vertex of $e_i$ which is not shared with $e_{i+1}$ to the vertex of $e_j$ which is not shared with $e_{j-1}$.
For certain values of $i$ and $j$ (detailed below), we will calculate the area of the subpolygon defined by the chord segment itself and the edges $e_i, e_{i+1}, \ldots, e_j$.

In order to query chords between two edges $e$ and $e'$ where the sequence of edges includes $e_n$ (i.e., the edge sequence is of the form $e, \ldots, e_{n-1}, e_{n}, e_{1}, e_{2}, \ldots e'$), we double the edge list to $e_1, e_2, \ldots, e_{n-1}, e_{n}, e_{1}', e_{2}', \ldots, e_{n-1}', e_n'$ where $e_i = e_i'$ for $1 \leq i \leq n$.
Any chord query between two edges $e_i$ and $e_j$ where $j < i$ is instead queried as $e_i$ and $e_j'$.

Every node in $T_P$ records the extremal edges of a chord, and the area of that chord. Given the edge list defining $P$, $T_P$ is constructed as follows. 
Split the edge list $E$ into halves, $E_l$ and $E_r$, and recursively construct $T_l$ and $T_r$, respectively. Initialize $T_P$ as a new tree node. If $|E| = 1$, then set $l(T_P) = r(T_P) = E$, set the area $a(T_P) = 0$, and return $T_P$. Otherwise, set $l(T_P) = l(T_l)$ and $r(T_P) = r(T_r)$. 
Let $v_m$ be the vertex shared by $E_l$ and $E_r$, and let $v_l$ and $v_r$ be the vertices at the opposite ends of $E_l$ and $E_r$, respectively, from $v_m$. Then, $a(T_P) = \Delta v_l v_m v_r + a(T_l) + a(T_r)$.

The resulting tree can be built with only $\BigOh{n}$ preprocessing.
Each node $t$ of $T_P$ records the labels of the left and right edges of $P$ that it spans, and the total area of the subpolygon between those edges.
Given two edge labels $i$ and $j$, we can query $T_P$ and identify $\BigOh{\log{n}}$ subtrees between those labels which correspond to precalculated subpolygons.
This will not give us the total area of the subpolygon between any arbitrary $e_i$ and $e_j$, but it does reduce the problem significantly, as the remaining area to be calculated is bounded by a path of only $\BigOh{\log{n}}$ chords rather than $\BigOh{n}$ edges of $P$.

\begin{figure}[t]
\begin{center}
  \includegraphics[width=0.80\textwidth]{figures/fig_cvp_prep}
  \caption[Preprocessing the area of chords of $P$.]{Preprocessing the area of chords of $P$.  Each level forms a triangle with the chords of the previous level.}
  \label{fig:convexp:preprocessing}
\end{center}
\end{figure}


%------------------------------------------------------------------------------
\subsection{Locating Intersections}
\label{:convexp:intersections}

Following from the convexity of $P$, there can be at most $8$ intersections between $P$ and $Q$.
Each intersection can be found in the following way. 
Let $b_\alpha$ be the vertical boundary of $Q$ with $x$-coordinate at $\alpha$. 
Let $\iline{\alpha}$ be the vertical line through $b_\alpha$. 
Using a binary search on the edges of $P$, we can identify the edges $e_\alpha$ and $e_\alpha'$ which intersect $\iline{\alpha}$, if they exist, such that $e_\alpha$ is clockwise from $e_\alpha'$ along $P$.
Specifically, we test only the top chain of $P$ to find $e_\alpha$ and only the bottom chain of $P$ to find $e_\alpha'$.
We then determine if $e_\alpha$ and $e_\alpha'$ are intersected by the segment $b_\alpha$ itself by checking the $y$-coordinate of the intersection point. 
We will store a `nil' value with $e_\alpha$ and/or $e_\alpha'$ if the intersections do not exist or do not intersect $b_\alpha$.

Similarly, let $b_\gamma$ be the vertical boundary of $Q$ with $x$-coordinate at $\gamma$, and let $b_\beta$ and $b_\delta$ be the horizontal boundaries of $Q$ with $y$-coordinates at $\beta$ and $\delta$, respectively.  Using the same binary search technique on appropriate half-chains of $P$, we can collect $e_\beta$, $e_\beta'$, $e_\gamma$, $e_\gamma'$, $e_\delta$, and $e_\delta'$.  

If all of these values are nil, then $P$ does not intersect $Q$ at all, and one of the following is true:

\begin{itemize}
 \item $P \subset Q$, determined by selecting any edge of $P$ and checking whether it is contained in $Q$. This check takes $\BigOh{1}$ time.

 \item $Q \subset P$, determined by selecting any point $q \in Q$, projecting horizontal and vertical lines through $q$, and checking that the intersections of these lines with $P$ occur on opposite sides of $q$. This check takes $\BigOh{\log{n}}$ time.

 \item $P \cap Q = \emptyset$, determined to be true if the previous two tests are false.
\end{itemize}


% -----------------------------------------------------------------------------
\subsection{Chain Decomposition}

For the remainder of the query, we assume that $P$ and $Q$ \emph{do} intersect. 
Any intersections between them must come in pairs; i.e., if $P$ enters $Q$, it must leave elsewhere.
Our goal is to determine how $P$ and $Q$ interact, and then calculate the area of $P \cap Q$.  Figure \ref{fig:convexp:examples} shows some examples of how the two may interact.

\begin{figure}[t]
\begin{center}
  \includegraphics[width=1.00\textwidth]{figures/fig_cvp_examples}
  \caption{Examples of how $P$ and $Q$ may interact.}
  \label{fig:convexp:examples}
\end{center}
\end{figure}

Continuing to examine the problem in a clockwise direction, $P$ can enter $Q$ at $e_\alpha$, $e_\beta$, $e_\gamma$, and/or $e_\delta$.  Once $P$ enters $Q$, it must exit at one of $e_\alpha'$, $e_\beta'$, $e_\gamma'$, or $e_\delta'$ before any other entrance can occur.

We can find all entrance/exit pairs with the following steps.
\begin{enumerate}
 \item Consider the following clockwise ordering of the entrance and exit labels as a circular list: $e_\alpha$, $e_\delta'$, $e_\delta$, $e_\gamma'$, $e_\gamma$, $e_\beta'$, $e_\beta$, $e_\alpha'$.
 
 \item Starting anywhere in the list, scan for a non-nil entrance label (a non-prime edge).
 
 \item Find the subsequent non-nil exit label, by continuing to walk the list in circular order.
 
 \item Mark any corners of $Q$ which are bypassed by step 2 as outside of $P$. The corners which remain inside will be important later.
 
 \item We are finished when we visit a non-nil entry label for the second time.
\end{enumerate}

Figure \ref{fig:convexp:chord} shows an example. In the figure, $e_\beta$ is an edge that enters into $Q$, while $e_\delta'$ is the corresponding exit edge. Both $e_\alpha'$ and $e_\alpha$ are nil. The corners of $Q$ at $(\alpha, \beta)$ and $(\alpha, \delta)$ are outside of $P$ since we know $P$ entered $Q$ before $(\alpha, \beta)$, and therefore all corners of $Q$ must be outside until the next non-nil entrance label.

\begin{figure}[t]
\begin{center}
  \includegraphics[width=0.50\textwidth]{figures/fig_cvp_chord}
  \caption[A query $Q$ intersecting $P$ at points $s$ and $t$.]{A query $Q$ intersecting $P$ at points $s$ and $t$. The segment $st$ is a chord of $P$ through the interior of $Q$.}
  \label{fig:convexp:chord}
\end{center}
\end{figure}

For an entrance edge $e_i$ and exit edge $e_j$, let $s$ and $t$ be the intersection points of $e_i$ and $e_j$ with the boundary of $Q$, respectively. 
The segment $st$ partitions $Q$ into two parts and forms a chord of $P$.

Let $P_{st}$ be the subpolygon ``cap'' of $P$ defined by the segment $st$, the portion of $e_i$ from $s$ towards $e_{i+1}$, the edges $e_{i+1}, e_{i+2}, ..., e_{j-1}$, and the portion of $e_j$ from $e_{j-1}$ to $t$.
Since $P_{st} \subset Q$, we need to calculate its area.

We first check if $e_i = e_j$.
In that case, a single edge of $P$ has formed both the entrance and exit of $Q$, the chord $st$ is a subsegment of that edge, and the area of $P_{st}$ is 0.
Otherwise, $P_{st}$ is a convex polygon.
Using the preprocessed subpolygon areas stored in $T_P$, the chain of edges $e_{i+1}$ to $e_{j-1}$ can be decomposed into $\BigOh{\log{n}}$ preprocessed subpolygons whose total area is found in the tree.
The remaining region of $P_{st}$ for which we do not yet know the area is comprised of a path of $\BigOh{\log{n}}$ chords from $T_P$, the partial edges of $P_{st}$ up to $s$ and $t$, and the segment $st$.
This region is a convex polygon with $\BigOh{\log{n}}$ vertices and its area can be calculated in $\BigOh{\log{n}}$ time.

The above process is repeated for each entry and exit pair of intersections, giving at most four caps of $P$.
We still need to calculate the area to the inside of each chord (the side away from the chord's cap).
This interior area may also be defined in part by the edges and corners of $Q$. 

More precisely, let $P_I$ be the polygon defined by all of the intersection  points of $P$ with the boundary of $Q$ (e.g., the points $s$ and $t$ for each cap), as well as any corners of $Q$ which are inside $P$. $P_I \subset Q$ and is comprised of $\BigOh{1}$ vertices, all of which have been previously identified in earlier steps of the query. Its area can be calculated in $\BigOh{1}$ additional time.


%------------------------------------------------------------------------------
\subsection{Analysis and Remarks}
\label{:convexp:analysis}

Our method has a straight-forward analysis. 
Preprocessing is limited to the creation of $T_P$ which requires total preprocessing time of $\BigOh{n}$, since the merge step of the recursion requires only $\BigOh{1}$.
Our query method involves several steps, but each requires only $\BigOh{1}$ or $\BigOh{\log{n}}$ time. 
The following theorem and corollary summarize our overall approach.

\begin{theorem}
\label{th:convexp:area}
Let $P$ be a convex polygon consisting of $n$ edges. In $\BigOh{n}$ time and space, we can create a data structure which allows us to determine $\area{Q \cap P}$ in $\BigOh{\log{n}}$ time, for any axis-parallel rectangular query region $Q$.
\end{theorem}

\begin{corollary}
\label{cor:convexp:mp}
Let $P$ be a convex polygon consisting of $n$ edges, and let $0 < \rho \leq 1$ be a fixed parameter. With $\BigOh{n}$ time and space, we can determine if $\area{Q \cap P} \geq \rho \cdot \area{P}$ in $\BigOh{\log{n}}$ time for any axis-parallel rectangular query $Q$.
\end{corollary}

Our method is easily extended to allow a query rectangle with any orientation.
Nearly all of the steps involved in calculating the area are only concerned with edges of $P$ and intersection points of lines through the boundaries of $Q$.
The only place we use the axis-parallel property of $Q$ is in Section~\ref{:convexp:intersections} where we calculate those intersection points.

To allow $Q$ to be arbitrarily-oriented, we modify our intersection testing in the following way. 
During preprocessing, we calculate $c$, the centrepoint of $P$, which can be done in $\BigOh{n}$ time.
Let $\iline{l}$ be a line through any boundary of $Q$ and let $\iline{c}$ be the line perpendicular to $\iline{l}$ through the centrepoint of $P$.
The line $\iline{c}$ crosses the boundary of $P$ in two places, and, using a binary search on the edges of $P$, we can identify the edges $e_a$ and $e_b$ that $\iline{c}$ crosses.
These two edges divide $P$ into two chains, $E_{ab}$ and $E_{ba}$.
We can now perform a binary search on each of these chains looking for intersections with $\iline{l}$ (which may or may not cross $P$ at all).
Figure~\ref{fig:convexp:ao} illustrates this process.

\begin{figure}[t]
\begin{center}
  \includegraphics[width=0.60\textwidth]{figures/fig_cvp_ao}
  \caption{Locating the intersections of $P$ with an arbitrarily-oriented $Q$.}
  \label{fig:convexp:ao}
\end{center}
\end{figure}

Our method can also be extended to allow for querying by an arbitrary convex $k$-gon.
The number of intersections we have to locate increases to $\BigOh{k}$, as does the number of chords across $P$.
Otherwise, we use exactly the same preprocessing and general query method as for rectangles. 
We require $\BigOh{k \log{n}}$ time to find all intersection points between $P$ and $Q$ and to calculate the areas of each cap of $P$. 
A further $\BigOh{k}$ time is needed to calculate the area of $P_I$. 
The total query time for a convex $k$-gon is thus $\BigOh{k \log{n}}$.
The overall approach is summarized by the following corollary.

\begin{corollary}
\label{cor:convexp:karea}
Let $P$ be a convex polygon consisting of $n$ edges. In $\BigOh{n}$ time and space, we can create a data structure which allows us to determine $\area{Q \cap P}$ in $\BigOh{k \log{n}}$ time, for any convex $k$-gon query region $Q$.
\end{corollary}

Work by Czyzowicz \textit{et al.}\cite{DBLP:conf/cccg/CzyzowiczCU98} and then Iacono and Langerman\cite{IaconoL00} gives an alternate method for calculating the area of a convex polygon cut by a chord.
This method requires only $\BigOh{n}$ preprocessing time and space, and can answer area queries in $\BigOh{1}$ time.
We could use this method to reduce the time needed to find the area within each cap of $P$ to $\BigOh{1}$ with no increase in the cost of our preprocessing.
However, calculating the area of a rectangle still requires us to expend $\BigOh{\log{n}}$ time locating the points of intersection between $P$ and $Q$, so our overall query time remains unchanged.

\input{monotonep}
\section{Conclusion}
\label{conclusion}

In this paper, we introduce the \emph{partial enclosure range searching 
problem}.
Two variants of the problem are studied. In the first variant, 
a set of line segments $S$ is preprocessed so that the partial enclosure range
query for a query range $Q$ can be performed efficiently. In the second 
variant, 
$S$ is a polygon and $Q$ is an axis-parallel rectangle, and the objective of 
the 
\emph{partial enclosure area problem} is to compute the area of $S \cap Q$. 

When $S$ is a monotone polygon, our presented algorithm requires $O(n\log n)$ 
preprocessing 
time and space and the query time is $O(\log n)$. In \cite{BINT}, it is shown 
that the space can be 
improved to $O(n)$ by increasing the query time to $O(\sqrt{n})$. It is also 
shown in \cite{BINT} that 
if $S$ is a convex polygon then in $\BigOh{n}$ time and space, we can create a 
data structure which 
can compute the area of $S \cap Q$ in $\BigOh{\log{n}}$ time, for any arbitrary 
oriented rectangular 
query range $Q$. For the case where $S$ is a simple polygon, we can handle the 
queries where $Q$ is an 
axis-parallel slab. Unfortunately, we cannot extend our method for rectangular 
queries to work with simple polygons so easily.
While the multi-region formulas themselves do not use the monotone property, our 
tree of multi-region formulas does. 
The tree functions by partitioning the trapezoidal regions with respect to 
vertical lines, however, in a simple polygon, a vertical line passing through 
the boundary of one region may pass through the interior of another. 
This lack of clean partitioning prevents the multi-region formula from working 
correctly for all possible horizontal query lines which may be given as input to 
the formula. Thus, the partial enclosure problem for simple polygons is worth studying. 

\bibliographystyle{abbrv}
\bibliography{references}
\end{document}










