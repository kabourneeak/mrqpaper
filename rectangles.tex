\section{PERS problem for Axis-Parallel Rectangles}
\label{:rectangles}


We will present two variations of the \PERS{} problem in this section. 
In the first variation (Subsection~\ref{:rectangles:ap}), the line 
segments in $S$ will  be axis-parallel, whereas in the next variation 
(Subsection~\ref{:rectangles:ao}), we allow the segments in $S$ of 
arbitrary orientation. In both the variations, the query rectangle $Q$ 
is axis-parallel. 

%------------------------------------------------------------------------------
\subsection{Axis-Parallel Segments}
\label{:rectangles:ap}

We start our discussion with only horizontal segments. The solution 
for vertical segments is identical. We also discuss how to combine 
the two approaches at the end of the section. We define the problem 
as follows.

\begin{definition}
A segment $s \in S$ is said to satisfy the {\em partial enclosure 
property} with respect to a query object $Q$ if and only if 
$|s \cap Q| \geq \rho \cdot |s|$, where $|x|$ denotes the length 
of the segment $x$.
\end{definition}

\begin{problem}
Given a set $S$ of $n$ axis-parallel line segments in the plane, 
and a fixed parameter $\rho$ ($0 < \rho \leq 1$), we want to 
identify those segments which are sufficiently enclosed by 
an axis-parallel query rectangle $Q$ so as to satisfy the 
partial enclosure property. 
\end{problem}

We use the following notations during our discussions. Each segment 
$s_i =[(a_i,b_i), \ell_i] \in S$, $1 \leq i \leq n$ is defined 
by its left or bottom endpoints $(a_i, b_i)$ depending on whether 
$s_i$ is horizontal or vertical, and its length $\ell_i$. 
Our query rectangle $Q$ is given by its bottom-left corner $(\alpha, 
\beta)$ and its top-right corner $(\gamma, \delta)$. \remove{Thus, its 
width is $w=\gamma-\alpha$.} We say that $s_i \in_\rho Q$ if and only 
if $s_i$ satisfies the partial enclosure property w.r.t. $Q$, 
otherwise $s \not \in_\rho Q$. 



\begin{figure}[t]
\centering
\includegraphics[width=1.00\columnwidth]{figures/fig_oo_cases}
\caption{An axis-parallel query on axis-parallel segments. Different cases 
of horizontal segments interacting with the query region are shown.}
\label{:fig:rectangles:ap:cases}
\end{figure}

Needless to mention, here we need to consider only the segments 
satisfying $\beta \leq b \leq \delta$. 
Figure~\ref{:fig:rectangles:ap:cases} illustrates several cases 
regarding how a horizontal segment $s \in S$ (satisfying $\beta 
\leq b \leq \delta$) may interact with $Q$. 
Cases $(1)$, $(2)$, and $(3)$ demonstrate the cases where $s$ is 
entirely left, entirely within, or entirely right of $Q$, 
respectively. Case $(4)$ considers the situation where $s$ crosses 
only the left boundary of $Q$ (i.e., $\alpha \leq a+\ell \leq \gamma$). 
Depending on the partial enclosure parameter $\rho$, we further 
subdivide case $(4)$ into subcases $(4a)$ if $s \in_\rho Q$, and 
$(4b)$ if $s \not \in_\rho Q$. Cases $(5a)$ and $(5b)$ are similar 
to cases $(4a)$ and $(4b)$, but with respect to $\gamma$. Specifically, 
$s$ falls into case $(5a)$ or $(5b)$ when $\alpha \leq a \leq \gamma$ 
and $a+\ell > \gamma$. In case $(6)$, $s$ crosses both the left and right 
boundaries of $Q$, and, crucially, neither of its endpoints are inside $Q$. 
Here, the subcases are $(6a)$ if $s \in_\rho Q$ and in$(6b)$ if $s \not 
\in_\rho Q$. Our goal is to identify all segments belonging to cases 
$(2)$, $(4a)$, $(5a)$, and $(6a)$, and none of the segments belonging 
to any other case.

Let $w = \gamma - \alpha$ be the width of $Q$. Thus, we need to consider 
only the segments in $S_1 = \{s_i \in S \st \beta \leq b_i \leq \delta 
~\&~ \ell_i \leq \frac{w}{\rho}\}$.  We partition the members in $S_1$ 
according the location of their left endpoint with respect to $\alpha$. 
Specifically, let $S_L = \{s_i \in S_1 \st a_i < \alpha\}$ and let 
$S_R = \{s_i \in S_1 \st a_i \geq \alpha\}$.
Now, we test an appropriate partial enclosure expression to determine 
whether $s_i$ should be counted. For segments in $S_L$, we want to 
ensure that ``not too much of $s_i$ is outside of $Q$'', i.e., $S_L' 
= \{ s_i \in S_L \st \alpha - a_i < (1 - \rho) \cdot l_i\}$. 
\complain{Please check whether it discards Case 6b also from the counting}
For segments in $S_R$, we want to ensure that ``enough of $s_i$ is inside 
$Q$'', i.e., $S_R' = \{s_i \in S_R \st \gamma - a_i \geq  \rho l_i\}$.

\begin{observation}
The subset of segments satisfying the partial enclosure property is 
$S_\rho = S_L' \cup S_R'$. 
\end{observation}

Thus, the points in $S_L'$ can be identified by mapping each 
segment $s_i\in S$ to a point $\hat{s_i}=(b_i,\rho\cdot\ell_i,
a_i,a_i + (1-\rho) \cdot\ell_i)$ in $I\!\!R^4$, and then observing  
those points lying inside the  four dimensional query box 
$\hat{Q}=[\beta, \delta] \times (0,w] \times (-\infty,\alpha] \times 
(\alpha,\infty)$. Similarly, the points in $S_R'$ can be identified 
by mapping each segment $s_i\in S$ to a point 
$\hat{\hat{s_i}}=(b_i,\rho\cdot\ell_i,a_i,a_i + \rho 
\cdot\ell_i)$ in $I\!\!R^4$, and then observing  
those points lying inside the  four dimensional query box 
$[\beta, \delta] \times (0, w] \times [\alpha, \infty) 
\times (-\infty, \gamma]$.

We can answer these queries by constructing two 4D range 
trees~\cite{Deberg} with two sets of points $\{\hat{s_i}|s_i 
\in S\}$ and $\{\hat{s_i}|s_i \in S\}$ respectively, and 
executing the counting query with the corresponding 4D query 
rectangle. The preprocessing time and space required for 
constructing these two range trees are both $\BigOh{n \log^3{n}}$, 
and the query can be answered in $\BigOh{\log^3{n}}$ time. Finally, 
we report the sum of two results as the answer of the query. 
Note that, we can use the same first three levels for both the 
range trees since the first three components of both the types 
of query points (for $a<\alpha$ and $a > \alpha$) are same. In the  
third level, we create \emph{two} associated structures, one for 
each of the partial enclosure expressions, and query each one as 
needed. 

For vertical segments, the method of querying is exactly similar 
to that for horizontal segments, only we need to consider the 
height of $Q$ instead of its width, and need to consider symmetric 
coordinates of each segment while mapping them to points in 4D. 
The following theorem summarizes the solution.

\begin{theorem}
\label{th:ap}
Given a set of $n$ axis-parallel line segments, we can identify a set of disjoint subsets containing all segments which satisfy the partial enclosure property for an axis-parallel query rectangle in $\BigOh{\log^3{n}}$ time, with a data structure requiring $\BigOh{n\log^3{n}}$ preprocessing time and space.
\end{theorem}

%------------------------------------------------------------------------------
\subsection{Arbitrarily-Oriented Segments}
\label{:rectangles:ao}

In this subsection, we allow each segment $s_i \in S$ to have any 
arbitrary orientation, and may intersect among themselves. We use 
$s_i= [(a_i,b_i), (c_i,d_i)]$ to denote a segment, where $(a_i,b_i)$ 
and $(c_i,d_i)$ are two end-points of $s_i$ satisfying $a_i\leq c_i$, 
and if $a_i = c_i$ then $b_i < d_i$. The problem statement is as 
follows.

\begin{problem}
Given a set of $n$ arbitrarily-oriented line segments in the plane, and a 
fixed parameter $\rho$ such that $1/2 < \rho \leq 1$, we want to identify 
those segments which are sufficiently enclosed by an axis-parallel query 
rectangle $Q$ so as to satisfy the partial enclosure property. A segment 
$s$ satisfies this property if and only if $|s \cap Q| \geq \rho \cdot |s|$.
\end{problem}



\subsubsection{Decomposing the Problem}
\label{:rectangles:ao:approach}

Here, we need three principal cases to consider: (i) those segments which 
have both endpoints inside $Q$, (ii) those with only one endpoint inside 
$Q$, and (iii) those with both endpoints outside $Q$ but they intersect $Q$.  

We first construct a 4D range tree $\cal T$ with the points $(a_i, b_i, c_i, d_i)$ 
for each $s_i \in S$, and execute a 4D range query with query box 
$[\alpha, \gamma] \times [\beta, \delta] \times [\alpha, \gamma] \times 
[\beta, \delta]$ corresponding to $Q=[\alpha,\beta] \times [\gamma,\delta]$ 
to identify all the segments satisfying the different cases. The points in Case (i)
will all satisfy partial enclosure property.

For the segments satisfying case (ii), we apply the same method as in 
case (i), but on a {\it virtual} segment as follows. For each $s_i=
[(a_i,b_i),(c_i,d_i)] \in S$, create two virtual segments $s_i'$ and 
$s_i''$, where $s_i' \subseteq s_i$, $s_i'$ has an endpoint $(a_i,b_i)$, 
and $|s_i'| = \rho \cdot |s_i|$, and $s_i'' \subseteq s_i$, $s_i''$ 
has an endpoint $(c_i,d_i)$, and $|s_i''| = \rho \cdot |s_i|$. Thus, if 
$s_i' \in Q$ and/or $s_i'' \in Q$, then $s_i \in_\rho Q$. Counting both 
cases may double-count some 
segment(s) which has both endpoints in $Q$, so we subtract the count 
obtained in Case (i). Using the 4D range query, we can solve this case 
also.

Case (iii), where both the endpoints of a segment are outside $Q$, 
is the most challenging case to handle. We begin by partitioning the 
space outside $Q$ into 8 regions by extending lines through its 
horizontal and vertical boundaries. These regions are labelled as 
$I, II, \ldots, VIII$ in anticlockwise order starting from the 
left-middle region (see Figure~\ref{fig:rectangles:ao:regions}). Any 
segment which passes through $Q$ but has its neither endpoint in $Q$ 
will have its endpoints in two distinct regions. Not every pair of 
regions is legal\footnote{A segment with its two endpoints 
in regions $I$ and $II$ respectively, cannot intersect $Q$.}.

\begin{figure}[t]
\begin{center}
  \includegraphics[width=0.5\textwidth]{figures/fig_ao_regions}
  \caption{The 8 regions surrounding $Q$.}
  \label{fig:rectangles:ao:regions}
\end{center}
\end{figure}

A segment which passes through $Q$ may involve in any one of the following 
four cases depending on the pair of regions containing its two endpoints:

\begin{description}
\item[(A)] In two parallely opposite cells: $(I, V), (III, VII)$.
\item[(B)] In two non-corner cells adjacent to a corner of $Q$: $(I, III), (III, V), (V,VII), (VII, I)$.
\item[(C)] In a corner cell and a non-corner cell: $(II, V)$, $(II, VII)$, $(IV, I)$, 
$(IV, VII)$, $(VI, I)$, \newline $(VI, III)$, $(VIII, III)$, $(VIII, V)$.
\item[(D)] In two corner cells: $(II, VI), (IV, VIII)$.
\end{description}

We will query for each class separately and combine our results at the end. 
In the first phase, we identify each class of problem 
by performing an appropriate rectangular range searching in the 
data structure $\cal T$. 
This also indicates which partial enclosure expressions we need to test in 
the second phase of the query. We develop expressions for each case below. 
In all the cases, we require the following additional definitions. Let 
$o_i = (e_i, f_i)$ be the mid-point of $s_i$. We use  
$d(u,v)$ to denote the Euclidean distance between two points $u,v \in 
I\!\!R^2$. 

\begin{figure}[h]
\begin{minipage}[b]{0.5\linewidth}
\centering
\includegraphics[width=0.80\textwidth]{figures/fig_ao_case1}\\
(a)
\end{minipage}
\begin{minipage}[b]{0.5\linewidth}
\centering
\includegraphics[width=0.80\textwidth]{figures/fig_ao_case2}\\
(b)
\end{minipage}
\caption{Demonstration of (a) Case A and (b) Case B}
\label{fig:rectangles:ao:case12}
\end{figure}



%------------------------------------------------------------------------------
{\bf Case (A):}
This case deals with the segments that cross the parallel sides of $Q$. 
We present the details of the partial enclosure property for the subcase 
where $p=(a,b) \in I$ and $q=(c,d) \in V$; the solution for the subcase 
$(III, VII)$ is similar.

Assume for now that $o \in Q$. Let 
$p'$ and $q'$ be the points of intersection of the segment $s$ with 
$Q$ closest to $p$  and $q$ respectively. Let us draw a right angle 
triangle $\Delta p u o$ whose edge $[u,o]$ intersects the boundary 
of $Q$ at $u'=(\alpha, f)$. Similarly, the edge $[v,o]$ of  
$\Delta q v o$  intersects the boundary of $Q$ at 
$v' = (\gamma, f)$ (see Figure~\ref{fig:rectangles:ao:case12}(a)). 
The segment $s$ satisfies partial enclosure property if 
$\frac{d(p, p') + d(q, q')}{d(p, q)} \leq 1 - \rho$. By similarity 
of triangles $\Delta puo$ and $\Delta p'u'o$, we have
$\frac{d(p, p')}{d(p, o)} = \frac{d(p, p')}{L} = 
\frac{d(u, u')}{d(u, o)} = \frac{2 d(u, u')}{2 d(u, o)} = 
\frac{2(\alpha - a)}{c - a}$. Similarly, from the similarity of 
triangles $\Delta qvo$ and $\Delta q'v'o$, we have
$\frac{d(q, q')}{d(q, o)} = \frac{2(c - \gamma)}{c - a}$.

Thus, $\frac{d(p,p')+d(q,q')}{d(p,q)}=\frac{d(p,p')+d(q,q')}{2L} = 
\frac{1}{2} \left ( \frac{d(p,p')}{L} + \frac{d(q,q')}{L} \right ) 
 = \frac{\alpha - a}{c - a} + \frac{c - \gamma}{c - a} 
= 1 + \frac{\alpha - \gamma}{c - a}$. 

Therefore, the inequality $\frac{d(p, p') + d(q, q')}{d(p, q)} 
\leq 1 - \rho$ can instead be evaluated as $1 + \frac{\alpha - 
\gamma}{c - a} \leq 1 - \rho$, which is based on two query 
variables $\alpha$ and $\gamma$. We can further simplify the 
expression to $\rho(c - a) \leq \gamma - \alpha$, where the value 
of $\gamma - \alpha$ is a single query variable expression 
calculated at query time. This inequality can therefore be checked 
by augmenting the search tree $\cal T$ with another level with 
respect to the variable $h=\rho(c-a)$ corresponding to the segments 
in $S$, and querying with $h \leq \gamma-\alpha$ at the relevant 
nodes in that level of $\cal T$. Note that, of $o \not\in Q$, then 
$s$ does not satisfy the partial enclosure property, and this test 
will also fail. So, no test is required to check whether $q \in Q$ 
or not.  



%------------------------------------------------------------------------------
{\bf Case (B)} This case is concerned with the segments which cross 
the mutually perpendicular sides of $Q$. We present the details 
of the partial enclosure property for the case where $p \in I$ 
and $q \in III$; the solutions for the other subcases  
are similar. 

Again let us assume that $o \in Q$\footnote{If $o \not\in Q$, $s$ 
will not satisfy the partial enclosure property, and it can be shown 
that test derived for this case will fail.}. As earlier, let $p'$ 
and $q'$ be the points of intersection of $s$ with the boundary of 
$Q$ closest to $p$ and $q$ respectively. As in Case (A), here also 
we draw the 
right-angle triangles $\Delta p u o$ and  $\Delta p'u'o$. In 
$\Delta p u o$, let $u' = (\alpha, f)$ be the point of intersection 
of the line segment $[u, o]$ with the boundary of $Q$, and 
$\Delta p'u'o$ is similar to $\Delta p u o$. Similarly, in 
$\Delta q v o$, the point $v' = (e, \beta)$ 
is the intersection point of $[v, o]$ with the boundary of $Q$, 
where $\Delta q' v' o$ is similar to $\Delta q v o$. See 
Figure~\ref{fig:rectangles:ao:case12}(b). We need to identify those 
segments satisfying $\frac{d(p, p')+d(q, q')}{d(p, q)} \leq 1-\rho$. 

By similarity of triangles, we have  $\frac{d(p, p')}{d(p, o)} = 
\frac{d(p, p')}{L} = \frac{d(u, u')}{d(u, o)} = \frac{2(\alpha - a)}
{c - a}$, and similarly, $\frac{d(q, q')}{d(q, o)} = \frac{2(\beta - d)}{b - d}$. 
Thus,
$\frac{d(p, p') + d(q, q')}{d(p, q)}=\frac{d(p, p') + d(q, q')}{2L} 
=\frac{1}{2} \left (\frac{d(p, p')}{L} + \frac{d(q, q')}{L} \right) 
= \frac{\alpha - a}{c - a} + \frac{\beta - d}{b - d}$.

Therefore, the inequality $\frac{d(p, p') + d(q, q')}{d(p, q)} 
\leq 1 - \rho$ can be tested by checking the equivalent inequality 
$\frac{\alpha - a}{c - a} + \frac{\beta - d}{b - d} \leq 1 - \rho$, 
defined on the two query variables $\alpha$ and $\beta$. On 
simplification, it becomes $\alpha + \beta \cdot \frac{c-a}{b-d} \leq 
\left ( (1 - \rho) + \frac{d}{b-d} \right ) \cdot (c-a) + a$.

Thus, we can map each segment $s$ satisfying Case (B) to a point  $(z,w)=
\left(\frac{c-a}{b-d}, \left ( (1 - \rho) + \frac{d}{b-d} \right ) \cdot 
(c-a) + a \right )$ in the (dual) plane. The segments matching the 
partial enclosure expression correspond to the points satisfying the 
half-plane $y \geq \beta x + \alpha$ in the dual plane. In order 
to perform this test, we need to add a half-plane query data structure
(ham-sandwich cut tree) \cite{chan2012} with the points $(z,w)$ 
corresponding to the 
points in $S$ at the leaf level of $\cal T$ (independent of the arrangement 
made in Case (A)), and querying with the halfplane obtained from the query 
interval $Q$ as the desired node in the leaf level of $\cal T$. 

%------------------------------------------------------------------------------
{\bf Case (C)}
In this Case, each segment have one endpoint in a corner region of $Q$, 
and the other endpoint is in a non-corner region of $Q$ as shown in Figure~\ref{fig:rectangles:ao:case3}. We present the details of the partial 
enclosure property for the subcase where $p \in I$ and $q \in IV$; the 
solutions for other subcases of this case are similar.

\begin{figure}[t]
\begin{center}
  \includegraphics[width=0.45\textwidth]{figures/fig_ao_case3}
  \hspace{1.0em}
  \includegraphics[width=0.45\textwidth]{figures/fig_ao_case3b}
  \caption{Example of segments in  subcase $(I, IV)$ of case 
  (C): the blue follows the proper handling, while the red shows the 
  incorrect handling.}
  \label{fig:rectangles:ao:case3}
\end{center}
\end{figure}

Consider the situation where $o \in Q$. We need to consider two 
subcases: (i) $s$ crosses the lines $x=\alpha$ and $y=\beta$ and 
(ii) $s$ crosses the lines $x=\alpha$ and $x=\gamma$. See 
Figures~\ref{fig:rectangles:ao:case3}(a) and 
\ref{fig:rectangles:ao:case3}(b) for examples. Subcase (i) is very 
close to the example presented for case (B). In that case, 
the only use of the fact that the endpoints of $s$ were located in 
regions $I$ and $III$ was to imply that $s$ crossed the lines 
$x=\alpha$ and $y=\beta$. As such, this subcase can use the same 
expression for testing the partial enclosing property for the segment 
$s$. Likewise, Subcase (ii) is similar to the 
example presented for case (A) and can use exactly that expression 
for testing $s$.

Our initial query for identifying the regions of $p$ and $q$ does 
not allow us to differentiate between the subcases. Instead, we will 
check both subcases simultaneously. From case (A) and case (B), 
we have the following expressions:
$ \frac{\alpha - a}{c - a} + \frac{c - \gamma}{c - a} \leq 1 - \rho$
 and
$\frac{\alpha - a}{c - a} + \frac{\beta - d}{b - d} \leq 1 - \rho$
respectively. If $s$ belongs to subcase (i), then 
$\frac{c - \gamma}{c - a} \leq \frac{\beta - d}{b - d}$ since 
$\gamma$ is farther right than the point where $s$ exits from $Q$. 
Likewise, in subcase (ii), 
$\frac{\beta - d}{b - d} \leq \frac{c - \gamma}{c - a}$.
Therefore, in either subcase, the result of the correct expression 
is larger than the incorrect one, allowing us to correctly reject 
segments by blindly checking both the conditions. This also holds when 
$o \not \in Q$, since the expression for at least one subcase would 
exclude the segment.


% -----------------------------------------------------------------------------
{\bf Case (D)}
This case is concerned with segments whose endpoints appear in 
diagonally opposite corner regions of $Q$. We present the details 
of the partial enclosure property for the case where $p \in II$ 
and $q \in VI$; the solution for the subcase $(VI, VIII)$ is similar.

\begin{figure}[t]
\begin{center}
  \includegraphics[width=0.45\textwidth]{figures/fig_ao_case4a}
  \hspace{1.0em}
  \includegraphics[width=0.45\textwidth]{figures/fig_ao_case4b}

  \vspace{2.0em}
  
  \includegraphics[width=0.45\textwidth]{figures/fig_ao_case4c}
  \hspace{1.0em}
  \includegraphics[width=0.45\textwidth]{figures/fig_ao_case4d}

  \caption{Example of segments of subcase case $(II, VI)$ of case (D): 
  in each figure, the blue lines show the proper handling.}
  \label{fig:rectangles:ao:case4}
\end{center}
\end{figure}

Assume for now that $o \in Q$. As in case (C), here also we need to 
consider several subcases depending on which sides of $Q$ are 
intersected by $s$, namely (i) $s$ crosses the lines $x=\alpha$ and 
$y=\delta$, (ii) $s$ crosses the lines $x=\alpha$ and $x=\gamma$, 
(iii) $s$ crosses the lines $y=\beta$ and $y=\delta$, and (iv)
$s$ crosses the lines $y=\beta$ and $x=\gamma$ (see Figure~
\ref{fig:rectangles:ao:case4}(a-d) for example of each subcase. 

Our query in the first phase for identifying the regions of $p$ and $q$ 
does not allow us to differentiate between subcases. However, 
it is sufficient to check all subcases simultaneously. The 
expression for subcase (ii) comes directly from case (A).
Using similar techniques as we did for cases (A) and (B), 
we develop the following set of partial enclosure expressions:

\begin{align*}
& (i) \quad \frac{\alpha - a}{c - a} + \frac{d - \delta}{d - b} \leq 1 - \rho
& (ii) \quad \frac{\alpha - a}{c - a} + \frac{c - \gamma}{c - a} \leq 1 - \rho \\
& (iii) \quad \frac{\beta  - b}{d - b} + \frac{d - \delta}{d - b} \leq 1 - \rho  
& (iv) \quad \frac{\beta  - b}{d - b} + \frac{c - \gamma}{c - a} \leq 1 - \rho \\
\end{align*}

Subcases (ii) and (iii) can be simplified to orthogonal range 
queries, while in subcases (i) and (iv) each segment needs to be mapped to 
a point in an appropriate plane and, as in case 
(B), the halfplane counting query needs to be performed for answering 
the query problem. 

To illustrate that checking all four conditions simultaneously 
yields correct results, consider what happens if $s$ belongs 
to subcase (i), where $s$ crosses $\alpha$ and $\delta$. In 
these circumstances, $u'$ is above $\beta$, and we have that 
$\frac{\alpha - a}{c - a} \geq \frac{\beta - b}{d - b}$.  
Likewise, $v'$ is left of $\gamma$, so $\frac{d - \delta}{d - b} 
\geq \frac{c - \gamma}{c - a}$.  Therefore, the expression for 
(i) dominates the other expressions (i.e., $\text{(a)} \geq 
\text{(b)}$, $\text{(a)} \geq \text{(c)}$, and $\text{(a)} \geq 
\text{(d)}$), allowing us to identify qualifying segments for 
subcase (i) by blindly checking all conditions. This property 
is true for segments belonging to subcases (ii), (iii), and (iv) 
as well. Furthermore, this test also holds when $o \not \in Q$, 
since the expression for at least one subcase would exclude the 
segment.


%------------------------------------------------------------------------------
\subsection{Construction and Analysis}
\label{:rectangles:ao:analysis}

Our method takes a very case-by-case approach to solving the problem, 
and executes in two phases. The first phase must classify segments 
belonging one of the interesting classes among (i) to (iv), and then 
its appropriate subclass by searching the data structure $\cal T$.  
In the second phase, the partial enclosure property is tested by 
querying in the respective augmented data structures of $\cal T$ in 
its last level. 

Broadly, our solution to this problem uses a multi-level range tree 
\cite{Deberg} for the classification phase, and a combination of 
range trees and canonical subsets structures \cite{chan2012} to 
check the partial enclosure expressions. 

Identifying the segments which satisfy each case is a matter of 
testing a set of several conditions, all of which must be true. 
The order that we test the conditions in makes no difference to 
the correctness of the algorithm, but can have an impact on its 
space requirements. We now summarize the total time and space 
complexity for solving this problem. 

As mentioned earlier, one can count the segments satisfying the
partial enclosure property among those belonging to the Cases (i) 
and (ii) by performing a 4D range query in the corresponding 
range tree $\cal T$. 

The sticks belonging Case (iii-A) and satisfying the partial enclosure 
property can be counted using 5D range searching with a rectangle $Q'=
(-\infty, \alpha) \times [\beta, \delta] \times (\gamma, \infty) 
\times [\beta, \delta] \times (-\infty, \gamma - \alpha]$ among the 
points $v_i = (a_i, b_i, c_i, d_i, \rho(c_i - a_i))$ corresponding 
to the segments $s_i \in S$. The following lemma summarizes 
the complexity results. 

\begin{lemma}
\label{lem:ao:class1:v}
Given a set $S$ of $n$ line segments, it can be preprocessed 
in $\BigOh{n\log^4{n}}$ preprocessing time and space, and one
can identify a subset of $S$ belonging to Case (iii-A) \emph{and} 
satisfying the partial enclosure property in $\BigOh{\log^4{n}}$ 
time.  
\end{lemma}

In case (iii-B)
\paragraph{Class (B) - subcase $(p, q) \in (I, III)$:} 
As with all cases, part of the problem involves first identifying those segments with endpoints in the appropriate regions.
In this case, however, the partial enclosure expression is evaluated using a half-plane query.

The classification portion is performed just as we have seen above. We map each segment to the 4D point $v_i = ( a_i, b_i, c_i, d_i )$.
The partial enclosure expression will be queried by the dual-space we developed in Section~\ref{:rectanges:ao:class2}. 
For each $s_i \in S$, we define $h_i = \left ( \frac{c - a}{b - d}, \left ( (1 - \rho) + \frac{d}{b-d} \right ) \cdot (c-a) + a \right )$.
We need to construct a data structure which can answer queries on pairs $(v_i, h_i)$.

By Theorem~\ref{th:rangetree}, we can query the classification component using a range tree according to the following lemma.
\begin{lemma}
\label{lem:ao:class2:v}
Given a set of $n$ line segments, we can identify a set of disjoint subsets containing all segments belonging to case (B) in $\BigOh{\log^3{n}}$ time using a data structure requiring $\BigOh{n\log^3{n}}$ preprocessing time and space.
\end{lemma}

By Theorem~\ref{th:chan}, we can query the half-plane component of our query objects using a canonical subsets data structure, giving the following lemma.

\begin{lemma}
\label{lem:ao:class2:h}
Given a set of $n$ line segments, we can identify a set of disjoint subsets containing all segments satisfying a half-plane representation of a partial enclosure expression in $\BigOh{\sqrt{n}\log{n}}$ time, using a data structure requiring $\BigOh{n\log{n}}$ preprocessing time and space.
\end{lemma}

Since the order that we check our conditions in does not affect correctness, we will check the half-plane condition first, then the endpoint classification.  
By Corollary~\ref{cor:multichan}, we can accomplish this by associating a classification structure with each subset of the half-plane structure. The resulting structure is summarized by the following lemma.

\begin{lemma}
\label{lem:ao:class2:c}
Given a set of $n$ line segments, we can identify a set of disjoint subsets containing all segments belonging to Case (B) \emph{and} which satisfy their partial enclosure property in $\BigOh{\sqrt{n}\log^4{n}}$ time using a data structure requiring $\BigOh{n\log^4{n}}$ preprocessing time and space.
\end{lemma}


\paragraph{Case (C) - subcase $(p, q) \in (I, IV)$:} 
This case can use the same orthogonal structure that we developed in Lemma~\ref{lem:ao:class1:v} and the same half-plane expression as in Lemma~\ref{lem:ao:class2:h}. 
One component of the query box needs to be updated to account for region IV, giving the following.
\[
(-\infty, \alpha) \times [\beta, \delta] \times (\gamma, \infty) \times (-\infty, \beta) \times (-\infty, \gamma - \alpha]
\]

By Corollary~\ref{cor:multichan}, we can combine these two data structures to create a new structure which can answer this case as summarized by the following lemma.

\begin{lemma}
\label{lem:ao:class3:c}
Given a set of $n$ line segments, we can identify a set of disjoint subsets containing all segments belonging to class (iii) \emph{and} which satisfy their partial enclosure property in $\BigOh{\sqrt{n}\log^5{n}}$ time using a data structure requiring $\BigOh{n\log^5{n}}$ preprocessing time and space.
\end{lemma}


\paragraph{Case (D) - subcase $(p, q) \in (II, VI)$:} 
This case has the largest number of partial enclosure expressions which need checking, in addition to the usual endpoint classification step. 
As a result, it will be the largest data structure to build.

The basic steps are just as in the last three cases. 
To cover endpoint classification and the two orthogonal partial enclosure expressions, we will use a data structure and query box similar to Lemma~\ref{lem:ao:class1:v}, but extended to 6D to cover the extra partial enclosure expression.
We then apply Lemma~\ref{lem:ao:class2:h} and Corollary~\ref{cor:multichan} twice to handle the two half-plane partial enclosure expressions and associate the orthogonal range tree. 
The entire structure for this case is summarized by the following lemma.

\begin{lemma}
\label{lem:ao:class4:c}
Given a set of $n$ line segments, we can identify a set of disjoint subsets containing all segments belonging to Case (D) \emph{and} which satisfy their partial enclosure property in $\BigOh{\sqrt{n}\log^7{n}}$ time using a data structure requiring $\BigOh{n\log^7{n}}$ preprocessing time and space.
\end{lemma}


\paragraph{Combining the Steps.} 

Querying the entire environment requires us to create the structures for handling the cases where one or both endpoints of a line segment lie entirely inside a query $Q$, as well as the structures for each of the cases when both endpoints lie outside of $Q$. 
Overall, this process is dominated by the structure required for the case (D) segments. 
The following theorem summarizes the overall solution.

\begin{theorem}
\label{th:ao}
Given a set of $n$ arbitrarily-oriented line segments, we can identify a set of disjoint subsets containing all segments which satisfy the partial enclosure property for an axis-parallel query rectangle in $\BigOh{\sqrt{n}\log^7{n}}$ time, using a data structure requiring $\BigOh{n\log^7{n}}$ preprocessing time and space.
\end{theorem}


