\section{PERS problem for Arbitrarily-Oriented Slabs}
\label{:slabs}

In this section, we show how we can answer partial enclosure range searching 
queries
where the elements of $S$ are horizontal line segments, and the query region 
$Q$ 
is a slab bounded by two parallel lines of arbitrarily orientation. Next, we 
extend 
our solution for a query region which is the intersection of two slabs. 

%------------------------------------------------------------------------------
\subsection{Querying with One Slab}
\label{:slabs:one}
In this section, we address the following problem. 
\begin{problem}
We are given a set $S$ of $n$ horizontal line segments in the 
plane, and a fixed parameter $\rho$ such that $0 < \rho \leq 1$. 
The objective is to identify those segments which are sufficiently 
enclosed inside (satisfy partial enclosure property with respect 
to) an arbitrarily oriented slab $Q$.
\end{problem}

As in Section \ref{:rectangles:ap}, here also we use a triple 
$s_i=(a_i,b_i,\ell_i)$ to represent an object in $S$. The 
query slab $Q$ is given by three inputs - $(\alpha, \beta, w)$,  
where the left and right bounding lines of $Q$ are defined by 
$L_1 : y = \alpha x + \beta$ and $L_2: y = \alpha x + \beta - 
\alpha w$ respectively. Thus, $w$ is the horizontal width of $Q$.

%------------------------------------------------------------------------------
\subsubsection{Identifying the Segments}
\label{:slabs:one:approach}
Identifying whether a segment $s_i \in_\rho Q$ (that is, whether $s_i$ satisfies the partial 
enclosure property w.r.t. $Q$), requires three broad steps:

\begin{enumerate}
\item Restrict segments to those which are ``not too long'' to fit 
sufficiently inside $Q$.
\item Classify all segments by whether their left endpoints are 
left or right of $L_1$.
\item For each class of segments, test an appropriate partial 
enclosure expression.
\end{enumerate}

We will use a multi-level canonical sets data structure 
\cite{chan2012} for answering these queries. 
We now describe the different steps of the query in more detail. 
Subsection~\ref{:slabs:one:analysis} describes the construction 
and analysis of the overall data structure.


%------------------------------------------------------------------------------
\subsubsection{Restrict Length}
%\label{:slabs:one:details:restrict}
The first step of the query is to perform a length test, as it 
simplifies future steps. With the query parameter $w$ given, 
only segments with length $\ell \leq \frac{w}{\rho}$ can satisfy 
the partial enclosure property. Thus, with a 1-dimensional 
orthogonal range to query we can extract a subset 
$S_1 = \{ s\in S \st \rho \ell \leq w \}$.


%------------------------------------------------------------------------------
\subsubsection{Classify Endpoints}
%\label{:slabs:one:details:classify}
The left endpoints of the members of $S$ can be in any one of the following three 
regions: (1) left of $L_1$, (2) between $L_1$ and $L_2$, and (3) right 
of $L_2$, however, only those segments belonging to cases (1) and (2) 
are interesting to us. We will see that partitioning segments as left 
or right of $L_1$ is sufficient, as we can discriminate between cases 
(2) and (3) while testing partial enclosure expressions in the next step.

Identifying segments whose left endpoints appear to the desired side of 
$L_1$ can be accomplished using a half-plane query on the left endpoints 
of the members in $S$. Thus, we have $S_L = \{ s \in S_1 \st p \text{ is 
left of } L_1 \}$, and let $S_R = \{ s \in S_1 \st p \text{ is right of 
} L_1 \}$.


%------------------------------------------------------------------------------
\subsubsection{Check the Partial Enclosure Property}
%\label{:slabs:one:details:pep}
For each of $S_L$ and $S_R$, the final step is to identify those segments 
which satisfy the partial enclosure property.

\begin{figure}[h]
\begin{minipage}[b]{0.5\linewidth}
\centering
\includegraphics[width=0.60\textwidth]{figures/fig_aoq_left_l1}\\
(a)
\end{minipage}
\begin{minipage}[b]{0.5\linewidth}
\centering
\includegraphics[width=0.60\textwidth]{figures/fig_aoq_left_l2}\\
(b)
\end{minipage}
\caption{(a) Segments whose left endpoints are left of $L_1$, (b) Segments whose 
left endpoints are between $L_1$ and $L_2$.}
\label{fig:slabs:one:aoq_left_l1_l1l2}
\end{figure}

Set $S_L$ can be classified into four subsets as follows (see Figure 
\ref{fig:slabs:one:aoq_left_l1_l1l2}(a)):

\begin{itemize}
 \item[(1)] Segments which are entirely left of $L_1$ (and which are not counted),
 \item[(2)] Segments which intersect $L_1$, and are sufficiently enclosed by $Q$,
 \item[(3)] Segments which intersect $L_1$, but not sufficiently enclosed by 
 $Q$ (and which are not counted),
 \item[(4)] Segments which intersect both $L_1$ and $L_2$. 
\end{itemize}

Given a segment $s \in S_L$, with left endpoint $p = (a,b)$, let $\iline{s}: y = b$ 
be the line through $s$, and let $(a', b)$ be the intersection point of $\iline{s}$ 
and $L_1$, where $a' = \frac{b - \beta}{\alpha}$. 

\begin{observation}\label{o21}
 $s \in_\rho Q$ if and only if $a' - a < (1 - \rho)\ell$.
\end{observation}
\begin{proof}
We look at each of the above cases to show that this single expression is 
enough 
to identify all segments correctly. 
First, the test directly identifies segments belonging to cases (2) or (3) 
since 
$a' - a$ is precisely the amount of $s$ outside of $Q$. 
This test rejects segments in case (1) since $a' - a > \ell
> (1-\rho)\ell$ and cannot satisfy partial enclosure property for any allowed 
value of $\rho$. 

Finally, this test is also sufficient for case (4) owing to the earlier length 
restriction step. If a segment is not too far left 
of $L_1$, then either it crosses only $L_1$, and case (2) holds, or it 
crosses $L_1$ and $L_2$. In the latter case we know that $|s| < 
\frac{w}{\rho}$, where $w$ is the width of the query slab. Since any 
segment in case (4) has the property $|s \cap Q| = w$, this implies that 
$s \in_\rho Q$.
\end{proof}

Thus, among the segments in $S_L$, we can count those satisfying the 
partial enclosure property by executing a halfplane range counting query:
$a' - a < (1 - \rho)\ell$ $\equiv$ $a + (1 - \rho)\ell > \frac{1}{\alpha} b - 
\frac{\beta}{\alpha}$. We map each segment $s$ to a point with coordinates 
$(b, a + (1-\rho)\ell)$. The segments in $S_L$ 
satisfying the partial enclosure expression then correspond to the points 
satisfying the half-plane $y > \frac{1}{\alpha}x - \frac{\beta}{\alpha}$.

%------------------------------------------------------------------------------

Similar to $S_L$, the segments in $S_R$ can be classified into four subsets, 
as shown in Figure \ref{fig:slabs:one:aoq_left_l1_l1l2}(b).



\begin{itemize}
 \item[(1)] Segments which are between $L_1$ and $L_2$. Here, $s \in_\rho Q$.
 \item[(2)] Segments whhich are entirely to the right of $L_2$. Here $s \cap Q = \emptyset$, 
 and hence $s \not \in_\rho Q$.
 \item[(3)] Segments which intersect $L_2$, and $s \in_\rho Q$.
 \item[(4)] Segments which intersect $L_2$, but $s \not \in_\rho Q$.
\end{itemize}
Let $\iline{s}$ be the horizontal line through $s$, and $(a'', b)$ be the 
intersection point of $\iline{s}$ with $L_2$ where $a'' = \frac{b - 
\beta}{\alpha}
 + w$. The following observation is easy to show.
\begin{observation} 
$s \in_\rho Q$ if and only if 
 $a'' - a \geq \rho \ell$.
\end{observation}
As in the case of $S_L$, here also the counting of elements in $S_R$ 
satisfying $a'' - a \geq \rho \ell$, or equivalantly $\rho \ell + a 
\leq \frac{1}{\alpha} b - \frac{\beta}{\alpha} + w$, can be done using 
a half-plane range query. Here we map each segment $s=(a,b,\ell)\in 
S_R$ to a point $(b, \rho \ell + a)$, and the query is performed with 
the halfplane  $y \leq \frac{1}{\alpha} x - \frac{\beta}{\alpha} + w$.

%------------------------------------------------------------------------------
\subsubsection{Construction and Analysis}
\label{:slabs:one:analysis}

We will use the multi-level canonical sets data structure described in 
Section~\ref{prelim} to perform parts of this query.
Each of the three steps given in subsection~\ref{:slabs:one:approach} will 
correspond to one nested level of the final data structure.
It is easiest to describe the structure inside-out, so we begin with the 
innermost structure.


%------------------------------------------------------------------------------

The innermost structure answers the length restriction step of the overall 
query. 
This is easily answered using a 1-dimensional range tree (AVL tree) 
\cite{Deberg}, 
keyed on the segment lengths. Thus, we have

\begin{lemma}
\label{lem:slabs:one:step1}
Given a set of $n$ horizontal line segments, we can identify a set of disjoint 
subsets containing all segments with length at most $\frac{w}{\rho}$ in 
$\BigOh{\log{n}}$ time, using a data structure of size $\BigOh{n}$, which can be 
built in $\BigOh{n \log{n}}$ preprocessing time.
\end{lemma}


% -----------------------------------------------------------------------------

To identify segments satisfying the partial enclosure property, we use the 
half-plane expressions as mentioned above. 
There are two expressions we need to test, one for $S_L$ and the other for $S_R$. 
By Theorem~\ref{th:chan}, we can answer this type of half-plane query directly 
using a canonical subsets data structure, yielding the following lemma 
applicable to both cases.

\begin{lemma}
\label{lem:slabs:one:step2a}
Given a set of $n$ horizontal line segments, we can identify a set of disjoint 
subsets containing all segments which are not ``too much'' to one side of a 
query line in $\BigOh{\sqrt{n}\log{n}}$ time, using a data structure of size 
$\BigOh{n\log{n}}$, which can be built in $\BigOh{n\log{n}}$ preprocessing time.
\end{lemma}

With each subset created for this structure, we will associate the structure 
required for Lemma~\ref{lem:slabs:one:step1}.
Applying Corollary~\ref{cor:multichan} gives us the following result.

\begin{lemma}
\label{lem:slabs:one:step2b}
Given a set of $n$ horizontal line segments, we can identify a set of disjoint 
subsets containing all segments which are not ``too much'' to one side of a 
query line \emph{and} which do not exceed a maximum length in 
$\BigOh{\sqrt{n}\log^2{n}}$ time, using a data structure of size 
$\BigOh{n\log{n}}$, which can be built in $\BigOh{n\log^2{n}}$ preprocessing 
time.
\end{lemma}


% -----------------------------------------------------------------------------

We need to classify the left-endpoints of the segments as left or right of 
$L_1$. 
By Theorem~\ref{th:chan}, each of these queries can be answered by a canonical 
subsets data structure.
With each subset created by this structure, we will associate the structure from 
Lemma~\ref{lem:slabs:one:step2b}. 
Applying Corollary~\ref{cor:multichan} gives us the following result.

\begin{lemma}
\label{lem:slabs:one:step3}
Given a set of $n$ horizontal line segments, we can identify a set of disjoint 
subsets containing all segments whose left endpoints are to one side of a line, 
which are not ``too much'' to one side of a query line, \emph{and} which do not 
exceed a maximum length in $\BigOh{\sqrt{n}\log^3{n}}$ time, using a data 
structure of size $\BigOh{n\log^2{n}}$, which can be built in 
$\BigOh{n\log^3{n}}$ preprocessing time.
\end{lemma}


% -----------------------------------------------------------------------------
Finally, to fully answer any query $Q$, we need to preprocess two such data structures, 
choosing our expressions appropriately for the left and right cases.
During query time, we will query both structures and combine their results.
The following theorem summarizes the overall solution.

\begin{theorem}
\label{th:slabs:one}
Given a set of $n$ horizontal line segments, we can identify a set of disjoint 
subsets containing all segments which satisfy the partial enclosure property for 
an arbitrarily-oriented query slab in $\BigOh{\sqrt{n}\log^3{n}}$ time, using a 
data structure of size $\BigOh{n\log^2{n}}$, requiring $\BigOh{n\log^3{n}}$ 
preprocessing time.
\end{theorem}


%------------------------------------------------------------------------------
%------------------------------------------------------------------------------
\subsection{Querying with Two Slabs}
\label{:slabs:two}

In this subsection, we consider a generalization of the slab query, where 
the objects in $S$ are same as in Section \ref{:slabs:one}, and the  
query input are a pair of slabs; we are interested in those segments 
which satisfy the partial enclosure property with respect to their intersection.
Formally, the problem is stated as follows:

\begin{problem}
Given a set $S$ of $n$ horizontal line segments in the plane and a fixed parameter 
$\rho$ such that $0 < \rho \leq 1$, we want to identify those segments $s \in S$ 
that satisfy 
the partial enclosure property ($s \in_\rho Q$) with respect to a parallelogram 
$Q$, or in other words,  $|s \cap Q| \geq \rho \cdot |s|$.
\end{problem}

If the two slabs are orthogonal, then the query is with respect to a 
rectangle of arbitrary orientation.  
The overall approach is very similar to subsection~\ref{:slabs:one}, and the 
data structure we will build to answer these queries is again based on the 
canonical sets structure outlined in Section~\ref{prelim}.



Here the query parallelogram $Q$ is defined by the intersection of two slabs, $S_p$ 
and $S_n$, which have positive and negative slopes, respectively. See Figure 
\ref{fig:slabs:two:ds}.

\begin{figure}[t]
\begin{center}
  \includegraphics[width=0.70\textwidth]{figures/fig_ds_slabs}
  \caption{A query parallelogram $Q$ formed by the inputs $\alpha$, $\beta$, 
$w_p$,
  $\gamma$, $\delta$, and $w_n$.}
  \label{fig:slabs:two:ds}
\end{center}
\end{figure}

Specifically, a query is given as a 6-tuple $(\alpha, \beta, w_p, \gamma, 
\delta, 
w_n)$, where $\alpha > 0$, $w_p > 0$, $\gamma < 0$, and $w_n > 0$. With these 
inputs, we define:

\begin{itemize}
 \item[$S_p$:] a slab $(\alpha,\beta,w_p)$ with positive slope $\alpha$, whose 
left edge is defined 
 by the line $L_1 : y = \alpha x + \beta$, and the width is $w_p$. Thus, its 
right edge 
 is defined by $L_2 : y = \alpha (x - w_p) + \beta$.
 \item[$S_n$:] a slab $(\gamma,\delta,w_n)$ with negative slope $\gamma$, whose 
left edge is defined 
 by the line $L_3 : y = \gamma x + \delta$, and the width is $w_n$. Thus, its 
right edge 
 is defined by $L_4 : y = \gamma (x + w_n) + \delta$.
 \item[$Q$:] The parallelogram $S_p \cap S_n$.
\end{itemize}



% -----------------------------------------------------------------------------
\subsubsection{Identifying the Segments}
\label{:slabs:two:approach}

Just as in the single slab problem, identifying the segments $s_i \in_\rho Q$ 
is accomplished by classifying its endpoints by how they interact with 
the boundaries of $S_p$ and $S_n$ forming $Q$,  and then testing an appropriate 
partial enclosure expression. 

\begin{figure}[t]
\begin{center}
  \includegraphics[width=0.65\textwidth]{figures/fig_ds_wpwn}
  \caption[Decomposition of the query region $Q$.]{Decomposition of the query 
region $Q$. The orientation of $Q$ depends on the relative widths of the slabs 
which define it.}
  \label{fig:slabs:two:wpwn}
\end{center}
\end{figure}

% -----------------------------------------------------------------------------

The query $Q$ is decomposed into three regions, $Q_a$, $Q_b$, and $Q_c$, by 
extending horizontal lines through each vertex of $Q$.
$Q_a$ and $Q_c$ are triangular regions, while $Q_b$ is a parallelogram. 
$Q_a$ is always defined by $L_1$ on the left and $L_4$ on the right. 
Likewise, $Q_c$, is always defined by $L_3$ on the left and $L_2$ on the right. 
The definition of $Q_b$ depends on the overall orientation of $Q$, which depends 
on the relationship between $w_p$ and $w_n$.
Specifically, $Q_b$ is defined by $L_1$ and $L_2$ when $w_p < w_n$ and defined 
by $L_3$ and $L_4$ when $w_p > w_n$.
When $w_p = w_n$, $Q_b$ disappears.


\subsubsection{Endpoint Classification:}
Once we know the definitions of each region, we can proceed with endpoint 
classification.
The goal of this step is to identify which partial enclosure expression is 
appropriate to test with each line segment $s$ by considering the appropriate 
borders of $Q$ it interacts with. 
Each region has its left, center, and right classification zone, $Z_L$, $Z_C$, 
and $Z_R$, where the center zone is the query region itself.
Figure~\ref{fig:slabs:two:endpoint} gives an illustration; $Z_L$, $Z_C$, and 
$Z_R$ are coloured blue, green, and red, respectively.

\begin{figure}[t]
\begin{center}
  \includegraphics[width=0.80\textwidth]{figures/fig_ds_endpoint}
  \caption[Classification zones for $Q$.]{Classification zones for each region 
of $Q$. Each region has a left, center, and right zone, coloured blue, green, 
and red, respectively, where we may find segments which need further testing.}
  \label{fig:slabs:two:endpoint}
\end{center}
\end{figure}

For $Q_a$, $Z_L$ is aligned with the base of $Q_a$, and extends up $L_1$. 
$Z_R$ is also aligned with the base of $Q_a$, and extends up $L_4$.
Both of these zones are open away from $Q$.
The zones for $Q_c$ are symmetric, being aligned with the top of the region, and 
with $Z_L$ and $Z_R$ extending down $L_3$ and $L_2$, respectively.

For $Q_b$, the classification zones we use depend on the orientation of the 
region.
When $w_p < w_n$, $Z_L$ is aligned with the base of $Q_b$ and extends up $L_1$, 
while $Z_R$ is aligned with the top of $Q_b$ and extends down $L_2$.
Conversely, when $w_p > w_n$, $Z_L$ is aligned with the top of $Q_b$ and extends 
down $L_3$, while $Z_R$ is aligned with the bottom of $Q_b$ and extends up 
$L_4$.
The former case is what is illustrated in Figure~\ref{fig:slabs:two:endpoint}.
$Z_C$ is $Q_b$ itself, which we decompose into two triangular subzones so that 
all of our zones are triangular in nature.
During the query, any time that we consider $Z_C$ for $Q_b$, we will query both 
subzones and consider their union.

Any segment interacting with a particular query region must have its endpoints 
in one of 6 combinations of classification zones: $(Z_L, Z_L)$, $(Z_L, Z_C)$, 
$(Z_L, Z_R)$, $(Z_C,\allowbreak Z_C)$, $(Z_C, Z_R)$, or $(Z_R, Z_R)$. 
For a segment $s \in S$ with endpoints $p$ and $q$:
\begin{itemize}
\item If $p, q \in (Z_C, Z_C)$, then $s$ is entirely inside $Q$ and $s \in_\rho 
Q$.
\item If $p, q \in (Z_L, Z_L)$, or $p, q \in (Z_R, Z_R)$ then $s$ is entirely 
outside $Q$ and $s \not \in_\rho Q$.
\item Otherwise, $s$ crosses one or both boundaries of the query region and we 
need to test the appropriate partial enclosure expression.
\end{itemize}


\subsubsection{Partial Enclosure Property:}
We begin by examining $Q_a$ in detail.
Let $s \in S$ be a horizontal line segment and let $\iline{s}$ be the line 
through $s$, defined by the equation $y = b$.
Let $(a', b) = \iline{s} \cap L_1$ and $(a'', b) = \iline{s} \cap L_4$, be the 
intersection points of $\iline{s}$ with $L_1$ and $L_4$, respectively, then $a' 
= \frac{b - \beta}{\alpha}$ and $a'' = \frac{b - \delta}{\gamma} - w_n$.
We have three combinations of classification zones that need further testing.

For the $(Z_L, Z_C)$ case, we know that $s$ only crosses $L_1$.
We check that not too much of $s$ is outside of $Q_a$, giving the  partial 
enclosure expression: $a' - a < (1 - \rho)\ell$, or in other words, 
$b \frac{1}{\alpha} - \frac{\beta}{\alpha} < a + (1 - \rho)\ell$.

We can query for segments matching this expression by performing a 
half-plane query on an appropriate dual-space.
For this expression in particular, we map each segment $s$ to a dual-point with 
coordinates $(b, a + (1-\rho)l)$, and query with the half-plane $y > 
\frac{1}{\alpha} x - \frac{\beta}{\alpha}$.

For the $(Z_C, Z_R)$ case, we know that $s$ only crosses $L_4$. 
We check that enough of $s$ is inside $Q_a$, which gives the  partial enclosure 
expression: $a'' - a \geq \rho \ell$, or in other words, $b \cdot 
\frac{1}{\gamma} - \frac{\delta}{\gamma} - w_n \geq a + \rho \ell$.
 We can test this expression by mapping each segment $s$ to a 
dual-point with coordinates $(b, a + \rho l)$ and then querying with the 
half-plane $y \leq \frac{1}{\gamma} x - \frac{\delta}{\gamma} - w_n$.

Finally, for the $(Z_L, Z_R)$ case, we know that both endpoints of $s$ are 
outside of $Q_a$, so we only need to measure the width of $s \cap Q_a$.  
Specifically, we require that:
\[
\begin{split} 
a'' - a' &\geq \rho l \\
%
\frac{b - \delta}{\gamma} - w_n - \frac{b - \beta}{\alpha} &\geq \rho l \\
%
\frac{b}{\gamma} - \frac{\delta}{\gamma} - w_n - \frac{b}{\alpha} + 
\frac{\beta}{\alpha} &\geq \rho l \\
%
b \cdot \left ( \frac{1}{\gamma} - \frac{1}{\alpha} \right ) + \left ( 
\frac{\beta}{\alpha} - \frac{\delta}{\gamma} - w_n \right ) &\geq \rho l \\
%
\end{split}
\]

We can test this expression by mapping each segment $s$ to a 
dual-point with coordinates $(b, \rho l)$ and then querying with the following 
half-plane.
$
y \leq \left ( \frac{1}{\gamma} - \frac{1}{\alpha} \right ) \cdot x + \left ( 
\frac{\beta}{\alpha} - \frac{\delta}{\gamma} - w_n \right ).
$

Classification into the left and right zones is somewhat ``rough'', as 
the zone continues above $Q_a$ itself. 
This is not a problem in the $(Z_L, Z_C)$ and $(Z_C, Z_R)$ cases since one 
endpoint of $s$ is classified directly in the closed zone $Z_C$ and the segments 
are horizontal.
However, it can happen that a segment classified into $(Z_L, Z_R)$ is entirely 
above $Q_a$. 
In this case, since we are measuring $a'' - a'$, and $a'' < a'$ above the apex 
of $Q_a$, the expression will be negative and the segment will be rejected.

The partial enclosure expressions for $Q_b$ and $Q_c$ are developed using 
exactly the same reasoning as for $Q_a$, differing only by which of the lines 
$L_1$, $L_2$, $L_3$, and $L_4$ we use to define $a'$ and $a''$.


% -----------------------------------------------------------------------------
\subsubsection{Construction and Analysis}
\label{:slabs:two:analysis}

We will use a multi-level query structure for this problem, just as we did for 
the single slab query.
Each of the steps given in Section~\ref{:slabs:two:approach} will correspond to 
one nested level of the data structure.
It is easiest to describe the structure inside-out, so we begin with the 
innermost structure.

To identify segments satisfying a partial enclosure expression, we use the 
half-plane dual-spaces we developed in subsection~\ref{:slabs:two:approach}. 
By Theorem~\ref{th:chan}, we can answer such a half-plane query directly using a 
canonical subsets data structure, yielding the following lemma.

\begin{lemma}
\label{lem:slabs:two:step1}
Given a set of $n$ horizontal line segments, we can identify a set of disjoint 
subsets containing all segments which are not ``too much'' to one side of a 
query line in $\BigOh{\sqrt{n}\log{n}}$ time, using a data structure of size 
$\BigOh{n\log{n}}$, which can be built in $\BigOh{n\log{n}}$ time.
\end{lemma}

We need to be able to classify the right endpoints of our segments into any of 
the classification zones of each query region.
Each of these regions are triangular in nature and can be queried with a 
canonical subsets data structure. 
With each subset created by this structure, we will associate the structure from 
Lemma~\ref{lem:slabs:two:step1}. 
Applying Corollary~\ref{cor:multichan} gives us the following result.

\begin{lemma}
\label{lem:slabs:two:step2}
Given a set of $n$ horizontal line segments, we can identify a set of disjoint 
subsets containing all segments which have their right endpoint in a query 
triangle, and which are not ``too much'' to one side of a query line, in 
$\BigOh{\sqrt{n}\log^2{n}}$ time, using a data structure of size 
$\BigOh{n\log^2{n}}$, which can be built in $\BigOh{n\log^2{n}}$ time.
\end{lemma}


Next, we need to be able to classify the left endpoints of our segments into any of 
the classification zones as well, which will use another canonical subsets data 
structure.
With each subset of this structure, we continue to build up our multi-level 
structure by associating the structure from Lemma~\ref{lem:slabs:two:step2}. 
By applying Corollary~\ref{cor:multichan} again, we have the following.

\begin{lemma}
\label{lem:slabs:two:step3}
Given a set of $n$ horizontal line segments, we can identify a set of disjoint 
subsets containing all segments which have their left and right endpoints in 
given query triangles, and which are not ``too much'' to one side of a query 
line, in $\BigOh{\sqrt{n}\log^3{n}}$ time, using a data structure of size 
$\BigOh{n\log^3{n}}$, which can be built in $\BigOh{n\log^3{n}}$ time.
\end{lemma}


Finally, 
to fully answer a query $Q$, we only need one instance of the structures for 
classifying the left and right endpoints, since these structures can support all 
of the different triangular classification queries we need to perform.
We will need to preprocess several versions of the innermost level which 
determines the partial enclosure expression, however.
During query time, we check $(Z_C, Z_C)$, $(Z_L, Z_C)$, $(Z_C, Z_R)$ and $(Z_L, 
Z_R)$ for every query region. 
For any of the latter three zone combinations which are non-empty, we check the 
innermost structure corresponding to the appropriate partial enclosure 
expression.
The following theorem summarizes this process.

\begin{theorem}
\label{th:slabs:two}
Given a set of $n$ horizontal line segments, we can identify a set of disjoint 
subsets containing all segments which satisfy the partial enclosure property 
with respect to the intersection of two query slabs in 
$\BigOh{\sqrt{n}\log^3{n}}$ time, using a data structure of size 
$\BigOh{n\log^3{n}}$, requiring $\BigOh{n\log^3{n}}$ preprocessing time.
\end{theorem}


%------------------------------------------------------------------------------
%------------------------------------------------------------------------------
\subsection{Remarks on Arbitrarily-Oriented Segments}
\label{:slabs:remarks}

Our general approach can be extended to identify arbitrarily-oriented line 
segments which satisfy the partial enclosure property for one or two slabs, 
however, the resulting structure is very ``case heavy'' and involves a large 
number of query variables.

Arbitrarily-oriented segments can cross through the borders of $Q$ in many more 
ways than horizontal ones.
This prevents us from partitioning the environment into a simple set of 
classification zones, and increases the number of cases to consider.

The partial enclosure expressions for each case will also involve higher degree 
polynomials.
For example, a segment $s$ which crosses $L_1$ and $L_2$ has intersection points 
$p' = (a', b')$ with $L_1$ and $q' = (a'', b'')$ with $L_2$.  
To consider how much of the segment is inside the slab, we need to calculate the 
length of the segment from $p'$ to $q'$, as follows.
\[
\begin{split}
\|p'q'\|^2 
&= \left( \sqrt{(a'' - a')^2 + (b'' - b')^2} \right)^2 \\
%
&= (a'' - a')^2 + (b'' - b')^2 \\
%
&= (a'')^2 - 2a'a'' + (a')^2 + (b'')^2 - 2b'b'' + (b')^2
\end{split}
\]

Each term in the above expression represents several query variables when we 
consider the actual values for $a'$, $b'$, $a''$, and $b''$.  
Let $\iline{s}$ be the line through $s$, and assume that it is defined as $y = 
mx + t$, then
$a'  = \frac{\beta - t}{m - \alpha}$, 
$b'  = m \frac{\beta - t}{m - \alpha} + t$, 
$a'' = \frac{\beta - t - \alpha w}{m - \alpha}$ and  
$b'' = m \frac{\beta - t - \alpha w}{m - \alpha} + t$.

Multiplying and squaring these expressions to calculate $\|p'q'\|^2$ results in 
a high degree polynomial with respect to the number of independent query 
variable expressions.
The situation worsens when we consider intersections between $L_1$ or $L_2$ with 
$L_3$ or $L_4$, as we have to consider $\gamma$ and $\delta$ as well.
Such an expression \emph{can} be mapped to a high-degree half-space and then be 
queried with a canonical subset structure, but the space and query times suffer.

