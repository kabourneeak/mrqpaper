\section{Preliminaries}
\label{prelim}

Let $P$ be a set of $n$ points in $I\!\!R^d$ ($d \geq 2$). A range tree for $P$ 
is a data structure that supports counting and reporting the members of $P$ in 
an axis-parallel rectangular query range
\cite[Chapter~5]{Deberg}.
It can be constructed in  $\BigOh{n \log^{d-1}{n}}$ time using 
$\BigOh{n \log^{d-1}{n}}$ space. The counting and reporting query time 
complexities are 
$\BigOh{\log^{d-1}{n}}$ and $\BigOh{\log^{d-1}{n} + k}$ respectively, where  $k$ 
is the number of reported points.

The range trees are used to query ranges which can be expressed by a single 
query variable expression.
In this paper, in some cases our ranges will take the form of half-planes 
comprised of two query variables, 
and for those, we use the following structure described in Chan\cite{chan2012}, 
restated here with $d=2$.

\begin{theorem}[Corollary 7.3, part $i$, in Chan\cite{chan2012}]
\label{th:chan}
Consider a set $P$ of $n$ points in the plane. We can form $\BigOh{n}$ canonical subsets of total size $\BigOh{n \log{n}}$ in 
$\BigOh{n \log{n}}$ time, such that the subset of all points inside any query 
simplex can be reported as a union of disjoint canonical subsets $C_i$ with 
$\sum_i{\sqrt{|C_i|}} \leq \BigOh{\sqrt{n}\log{n}}$ in time 
$\BigOh{\sqrt{n}\log{n}}$ with high probability with respect to $n$.
\end{theorem}

This structure is particularly well-suited to multi-part queries where we need 
to identify objects which satisfy a half-plane query, 
and which also satisfy some other arbitrary query condition for which we have an 
existing query method.
To support such a query, we construct a canonical subsets structure to perform 
the half-plane component of the query. 
Then, with the objects of each canonical subset, we construct secondary query 
structures.
A similar structure with slightly worse query time bounds by 
Matousek\cite{Matousek92} is presented in \cite[Chapter~16]{Deberg}, 
along with a method for using it to construct multi-part queries.

A multi-level query is executed by first querying the canonical subsets 
structure for all points satisfying the half-plane.
With each subset identified by the first step, we then evaluate the associated 
secondary structure.
The result is the set of all objects which satisfy both the half-plane query 
\emph{and} whatever conditions the second-level query may test.
If the secondary structure is itself a multi-level structure, then this 
procedure effectively adds one more layer, and we can repeat this process to any 
arbitrary number of levels.
The following corollary formalizes the use of canonical subset structures to 
build multi-level structures and details the preprocessing and query 
requirements.


\begin{corollary}
\label{cor:multichan}

Given a set of $n$ objects $a_1, a_2, \ldots, a_n$ which can be queried with a 
data structure $A$ requiring preprocessing space $S(n)$, 
preprocessing time $T(n)$, and query time $Q(n)$, and where each $a_i$ also has 
an associated point $p_i$ in the plane, we can construct a multi-level 
data structure which can identify all objects whose associated point $p_j$ 
satisfies a half-plane  \emph{and} where $a_j$ satisfies a query on $A$.

\begin{enumerate}
\item If $S(n) \in \BigOh{n \log^f{n}}$, $T(n) \in \BigOh{n \log^g{n}}$, and 
$Q(n) \in \BigOh{\sqrt{n} \log^h{n}}$, where $f, g, h \in \BigOh{1}, 0 \leq f 
\leq g$, then the resulting multi-level data structure requires 
$\BigOh{n\log^{f+1}{n}}$ preprocessing space, $\BigOh{n\log^{g+1}{n}}$ 
preprocessing time, and $\BigOh{\sqrt{n}\log^{h+1}{n}}$ query time with high 
probability.

\item If $S(n) \in \BigOh{n \log^f{n}}$, $T(n) \in \BigOh{n \log^g{n}}$, and 
$Q(n) \in \BigOh{\log^h{n}}$, where $f, g, h \in \BigOh{1}, 0 \leq f \leq g$, 
then the resulting multi-level data structure requires $\BigOh{n\log^{f+1}{n}}$ 
preprocessing space, $\BigOh{n\log^{g+1}{n}}$ preprocessing time, and 
$\BigOh{\sqrt{n}\log^{h+1}{n}}$ query time with high probabily.

\end{enumerate}
\end{corollary}

\begin{proof}
The preprocessing space of the multi-level structure requires $\BigOh{n 
\log{n}}$ space for the canonical subsets structure itself. In addition, the 
associate structures need 
$\sum_{i=1}^k{S(|C_i|)}$ space, which is less than or equal to 
$\sum_{i=1}^k{\BigOh{|C_i| \log^f{|C_i|}}}$ 
$\leq \BigOh{\log^f{n} \cdot \sum_{i=1}^k{|C_i|}}$ (since $|C_i| \leq n$ for all 
$i$)  
$\leq \BigOh{\log^f{n} \cdot n \log{n}}$  (by Theorem \ref{th:chan}). 
Thus, the total space complexity is $\BigOh{n\log^{f+1}{n}}$.
Preprocessing time is calculated in the same manner, resulting in the time 
complexity  of $\BigOh{n\log^{g+1}{n}}$.

Querying requires $\BigOh{\sqrt{n}\log{n}}$ time with high probability  to find 
the $k'$ disjoint canonical 
subsets representing the elements found by the top-level canonical subsets 
query, plus the time required to 
query the associated data structures.  

If $Q(n) \in \BigOh{\sqrt{n}\log^h{n}}$ then the total time to query the 
appropriate associated structures is
$\sum_{i=1}^{k'}{Q(|C_i|)}$
$\leq \sum_{i=1}^{k'}{\BigOh{\sqrt{|C_i|}\log^h{|C_i|}}}$
$\leq \BigOh{\log^h{n} \cdot \sum_{i=1}^{k'}{\sqrt{|C_i|}}}$
$\leq \BigOh{\log^h{n} \cdot \sqrt{n}\log{n}}$ (by Theorem \ref{th:chan}).

Otherwise, if $Q(n) \in \BigOh{\log^h{n}}$ then the total time to query the 
appropriate associated structures is
$\sum_{i=1}^{k'}{Q(|C_i|)}$
$\leq \sum_{i=1}^{k'}{\BigOh{\log^h{|C_i|}}}$
$\leq \BigOh{\sum_{i=1}^{k'}{\log^h{n}}}$
$\leq \BigOh{\log^h{n} \cdot \sum_{i=1}^{k'}{1}}$
$\leq \BigOh{\log^h{n} \cdot \sqrt{n}\log{n}}$ (by Theorem \ref{th:chan}).

Thus, the total query time  is $\BigOh{\sqrt{n}\log^{h+1}{n}}$ with high 
probability.
\end{proof}

We use this theorem and corollary as ``black boxes'' in Sections 
\ref{:rectangles} and \ref{:slabs}.